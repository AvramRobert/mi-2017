\chapter{Leitfaden // Programmakkreditierung // Siegel der Stiftung zur
Akkreditierung von Studiengängen in Deutschland (Akkreditierungsrat)}\label{Leitfaden // Programmakkreditierung // Siegel der Stiftung zur
Akkreditierung von Studiengängen in Deutschland (Akkreditierungsrat)}


\section{Vorschläge für die Bearbeitung und
Gliederung}\label{vorschluxe4ge-fuxfcr-die-bearbeitung-und-gliederung}

Die Durchführung eines Akkreditierungsverfahrens basiert auf der Vorlage
eines sogenannten Selbstberichtes seitens der antragstellenden
Hochschule.

Die Phase der Erstellung dieser Selbstbewertung bietet die Möglichkeit,
interne Qualitätssicherungs- (und Reflexions-)Prozesse zu nutzen, um
relevante Interessenträger einzubeziehen und Verbesserungspotentiale
freizusetzen. Im Idealfall wird das Akkreditierungsverfahren als Projekt
zur Qualitätsentwicklung in der Hochschule genutzt und nicht als formale
Prüfroutine durchlaufen.

Die Erstellung der Selbstbewertung setzt sich aus jeweils zwei Schritten
zusammen:

\begin{enumerate}
\def\labelenumi{\arabic{enumi}.}
\tightlist
\item
  \textbf{Selbstbewertung}: Im Selbstbericht bewertet die Hochschule in
  möglichst komprimierter Form selbst, ob und wie für den zu
  akkreditierenden Studiengang/die zu akkreditierenden Studiengänge die
  einzelnen Kriterien erfüllt sind und welche Besonderheiten ggf. zu
  berücksichtigen sind. Auch Abweichungen von den Kriterien können hier
  erläutert werden.
\end{enumerate}

Der Fokus liegt dabei vorrangig auf einer bewertenden, nicht
deskriptiven Einschätzung, die z.~B. nach Stärken, Schwächen,
Herausforderungen und Lösungen gegliedert sein kann. Die nachstehenden
„Leitfragen`` zu jedem Kriterium sollen dazu \emph{eine Hilfestellung}
bieten.

Die Selbstbewertung ist zugleich ein „Wegweiser`` durch ergänzende
Anhänge. Häufig reichen eine prägnante, kurz gefasste Einschätzung zu
dem jeweiligen Kriterium und ein Verweis auf einen Beleg im Anhang als
Dokumentationsgrundlage für das Akkreditierungsverfahren aus.

Richtet sich der Akkreditierungsantrag auf ein „Cluster`` inhaltlich
verwandter Studiengänge, sollten Informationen, die für alle
Studiengänge des Clusters gleichermaßen gelten, zusammengefasst werden.
Zugleich sollten studiengangspezifische Informationen (z. B. angestrebte
Lernergebnisse, Curriculum etc.) unterscheidbar ausgewiesen sein.

\begin{enumerate}
\def\labelenumi{\arabic{enumi}.}
\tightlist
\item
  \textbf{Evidenzen}: Es ist von zentraler Bedeutung, dass die
  vorgelegten Selbstbewertungen nachvollziehbar dokumentiert und durch
  geeignete Belege („Evidenzen``) untermauert werden. Dazu sollte ein
  Anhang mit entsprechenden Belegen („Evidenzen``) zusammengestellt
  werden. Dieser Anhang sammelt die internen Regelungen, Dokumente,
  quantitativen oder qualitativen Daten und Informationen, die bereits
  in der Hochschule vorliegen -- z. B. weil sie im Zuge der internen
  Qualitätssicherung ohnehin produziert werden und deshalb nicht eigens
  für das Akkreditierungsverfahren erstellt werden müssen. Eine
  \emph{exemplarische} Liste potentieller Nachweise, die nach Bedarf
  ergänzt oder abgeändert werden kann, findet sich als Hilfestellung in
  der nachfolgenden Gliederung der Selbstbewertung.
\end{enumerate}

Es empfiehlt sich, für die Erstellung der Selbstbewertung dieses
Gliederungsschema als Vorlage zu nutzen. Das Schema ist nach den
Akkreditierungskriterien aufgebaut und unterscheidet jeweils zwischen
Leitfragen für die Analyse und Hinweisen zu möglicherweise geeigneten
Evidenzen. Beide sind nicht verbindlich, sondern lediglich als
Hilfestellung gedacht.

Selbstbewertung und Evidenzen können je nach Digitalisierungsgrad des
hochschuleigenen Dokumenten- und Datenmanagements grundsätzlich
elektronisch aufbereitet sein, z.~B. auch Zugänge zu spezifischen
Webseiten, Datenbanken o. ä. enthalten. Je nach Bedarf einzelner
Gutachtergruppen bitten wir im konkreten Fall zusätzlich um die
Papierfassung der Antragsunterlagen, wobei mittelfristig der Übergang
zur ausschließlich elektronischen Dokumentation angestrebt wird.

%

\chapter{Qualifikationsziele des Studiengangskonzeptes}\label{Qualifikationsziele des Studiengangskonzeptes}


Das Studiengangskonzept orientiert sich an Qualifikationszielen. Diese
umfassen fachli-che und überfachliche Aspekte und beziehen sich
insbesondere auf die Bereiche

\begin{itemize}
\item
  wissenschaftliche oder künstlerische Befähigung,
\item
  Befähigung, eine qualifizierte Erwerbstätigkeit aufzunehmen,
\item
  Befähigung zum gesellschaftlichen Engagement
\item
  Persönlichkeitsentwicklung
\end{itemize}

\section{Leitfragen}\label{leitfragen}

\begin{itemize}
\item
  An welcher Stelle sind die jeweils im Kriterium genannten
  Kompetenz-Bereiche im Studiengang nach dem Verständnis der Hochschule
  abgebildet?
\item
  Wie wurde das angestrebte Kompetenzprofil des Studiengangs
  (weiter-)entwickelt (Auslöser, Vorgehen, Beteiligungen)?
\item
  Finden die definierten Kompetenzziele für Absolventen des
  Studienprogramms die Zustimmung von Lehrenden und Studierenden?
\item
  Wurde die Stimmigkeit der Lernziele des Studiengangs in den letzten
  Jahren überprüft? Aus welchen Gründen wurden ggf. Anpassungen
  vorgenommen?
\item
  Gibt es Auffälligkeiten bei den qualitativen oder quantitativen
  Daten/Informationen der Hochschule hinsichtlich der Akzeptanz des
  Kompetenzprofils auf dem Arbeitsmarkt?
\end{itemize}

\section{Mögliche Evidenzen}\label{muxf6gliche-evidenzen}

\begin{itemize}
\item
  Dokumente/Stellen, wo die Ziele und Lernergebnisse verankert u.
  veröffentlicht sind, z.B. Ordnungen, Homepage, Diploma Supplement,
  Studienführer
\item
  Interne Unterlagen, aus denen die Einbeziehung der verschiedenen
  Interessenträger hervorgeht, z.B. Vorgaben, Prozessbeschreibungen,
  Befragungsergebnisse, Protokolle
\item
  Ziele-Module-Matrix
\item
  Modulbeschreibungen, wie sie den Lehrenden und Studierenden zur
  Verfügung stehen
\end{itemize}

%

\chapter{Konzeptionelle Einordnung des Studiengangs in das Studiensystem}\label{Konzeptionelle Einordnung des Studiengangs in das Studiensystem}


Der Studiengang entspricht

(1) den Anforderungen des Qualifikationsrahmens für deutsche
Hochschulabschlüsse vom 21.04.2005 in der jeweils gültigen Fassung,

(2) den Anforderungen der Ländergemeinsamen Strukturvorgaben für die
Akkreditierung von Bachelor- und Masterstudiengängen vom 10.10.2003 in
der jeweils gültigen Fassung,

(3) landesspezifischen Strukturvorgaben für die Akkreditierung von
Bachelor- und Masterstudiengängen,

(4) der verbindlichen Auslegung und Zusammenfassung von (1) bis (3)
durch den Akkreditierungsrat.

\section{Leitfragen}\label{leitfragen}

\begin{itemize}
\item
  Inwieweit sehen die für den Studiengang Verantwortlichen die im
  Kriterium genannten Anforderungen (insbesondere ländergemeinsame und
  ggf. landesspezifische Strukturvorgaben) eingehalten? Wo sieht die
  Hochschule Abweichungen und wie sind diese begründet?
\item
  Auf welcher Berechnungsgrundlage fußt die Zuordnung von Kreditpunkten
  zu einzelnen Modulen?
\item
  Sind alle verbindlich vorgeschriebenen Studienbestandteile
  (einschließlich praktischer Studienphasen) kreditiert? Wenn nein,
  warum nicht?
\item
  Sind bei der Vergabe von Abschlusszeugnis und Diploma Supplement an
  die Studierenden Probleme bekannt geworden? Wenn ja, wie wurde darauf
  reagiert?
\end{itemize}

\section{Mögliche Evidenzen}\label{muxf6gliche-evidenzen}

\begin{itemize}
\item
  Studien-/Prüfungsordnung bzw. Zugangssatzung
\item
  Falls nicht in Ordnungen enthalten, ergänzende Dokumente, die
  Studienstruktur und -dauer, ggf. Studiengangsprofile, ggf. Einordnung
  in konsekutive oder weiterbildende Masterstudiengänge, Abschlüsse und
  Abschlussbezeichnungen belegen
\item
  Modulbeschreibungen, wie sie den Lehrenden und Studierenden zur
  Verfügung stehen
\item
  Dokumente, in denen Studienverläufe und deren Organisation geregelt
  sind (z.~B. Studienverlaufspläne)
\item
  Dokumente, die die Kreditpunktezuordnung hochschulweit /
  studiengangbezogen regeln
\item
  exemplarisches Zeugnis je Studiengang
\item
  exemplarisches Diploma Supplement je Studiengang
\item
  exemplarisches Transcript of Records je Studiengang
\end{itemize}

%

\chapter{Studiengangskonzept}\label{Studiengangskonzept}


Das Studiengangskonzept umfasst die Vermittlung von Fachwissen und
fachübergreifendem Wissen sowie von fachlichen, methodischen und
generischen Kompetenzen.

Es ist in der Kombination der einzelnen Module stimmig im Hinblick auf
formulierte Qualifikationsziele aufgebaut und sieht adäquate Lehr- und
Lernformen vor. Gegebenenfalls vorgesehene Praxisanteile werden so
ausgestaltet, dass Leistungspunkte (ECTS) erworben werden können.

Es legt die Zugangsvoraussetzungen und gegebenenfalls ein adäquates
Auswahlverfahren fest sowie Anerkennungsregeln für an anderen
Hochschulen erbrachte Leistungen gemäß der Lissabon Konvention und
außerhochschulisch erbrachte Leistungen. Dabei werden Regelungen zum
Nachteilsausgleich für Studierende mit Behinderung getroffen.
Gegebenenfalls vorgesehene Mobilitätsfenster werden curricular
eingebunden.

Die Studienorganisation gewährleistet die Umsetzung des
Studiengangskonzeptes.

\section{Leitfragen}\label{leitfragen}

\begin{itemize}
\item
  Welchen Beitrag leistet das Curriculum/leisten die einzelnen Module
  aus Sicht der für den Studiengang Verantwortlichen und Beteiligten zum
  Erreichen des angestrebten Kompetenzprofils?
\item
  Hat sich im Zuge eines Abgleichs von angestrebtem Kompetenzprofil und
  Curriculum in den letzten Jahren Anpassungsbedarf ergeben? Welche
  Gründe gab es dafür? Wie wurde reagiert?
\item
  Wie wird erreicht, dass die Module in sich stimmig sind, zueinander
  passen und wo nötig aufeinander aufbauen? Wie reagieren die für einen
  Studiengang Verantwortlichen, wenn einzelne Module sich nicht (mehr)
  in das Gesamtkonzept des Studiengangs einfügen?
\item
  Woran erkennen die für den Studiengang Verantwortlichen, dass die
  Module eines Studiengangs \emph{in ihrer Gesamtheit} das angestrebte
  akademische Niveau tragen?
\item
  Inwieweit tragen die angebotenen Wahlmöglichkeiten im Studiengang zum
  Erreichen des angestrebten Kompetenzprofils bei?
\item
  Woran erkennen die Lehrenden und die für den Studiengang
  Verantwortlichen, dass die gewählten didaktischen Instrumente und
  Methoden das Erreichen der Lernergebnisse des Studiengangs
  unterstützen?
\item
  Können alle Lehrenden die ihrer Meinung nach idealen didaktischen
  Instrumente und Methoden einsetzen? Wenn nein, warum nicht?
\item
  Welche Elemente unterstützen das eigenständige wissenschaftliche
  Arbeiten von Studierenden?
\item
  Erfüllen die ggf. in einem Studiengang vorgesehenen Praxisphasen die
  Erwartungen im Hinblick auf die angestrebten Lernergebnisse?
\item
  Welchen Prinzipien folgt die Hochschule im Umgang mit extern
  erworbenen Leistungen von Studierenden?
\end{itemize}

\section{Mögliche Evidenzen}\label{muxf6gliche-evidenzen}

\begin{itemize}
\item
  Curriculare Übersicht/Studienverlaufsplan, aus der/dem Semesterlage,
  Umfang und studentische Arbeitslast der Module pro Semester
  hervorgehen (ggf. mit Veröffentlichungsort wie z.~B. Homepage,
  Studienführer, Studien- bzw. Prüfungsordnungen) bzw. Dokumente, in
  denen Studienverläufe und deren Organisation geregelt sind
\item
  Dokumente, aus denen die geltenden Regelungen zur
  (Auslands-)Mobilität, Praxisphasen und Anerkennung von an anderen
  Hochschulen / außerhalb der Hochschule erbrachte Leistungen erkennbar
  sind
\item
  Ziele-Module-Matrix
\item
  Modulbeschreibungen, wie sie den Lehrenden und Studierenden zur
  Verfügung stehen
\item
  Dokumente aus dem täglichen Gebrauch an der Hochschule, aus denen das
  vorhandene Didaktik-Konzept hervorgeht
\item
  Einschlägige Ergebnisse interner Befragungen und Evaluationen
\item
  Ggf. Daten zur (Auslands-)Mobilität von Studierenden und zu
  Praxiseinsätzen von Studierenden
\item
  Informationen über die Profile der Bewerber und der zugelassenen
  Studierenden
\end{itemize}

%

\chapter{Studierbarkeit}\label{Studierbarkeit}


Die Studierbarkeit des Studiengangs wird gewährleistet durch:

\begin{itemize}
\item
  die Berücksichtigung der erwarteten Eingangsqualifikationen,
\item
  eine geeignete Studienplangestaltung
\item
  die auf Plausibilität hin überprüfte (bzw. im Falle der
  Erstakkreditierung nach Er-fahrungswerten geschätzte) Angabe der
  studentischen Arbeitsbelastung,
\item
  eine adäquate und belastungsangemessene Prüfungsdichte und
  -organisation,
\item
  entsprechende Betreuungsangebote sowie
\item
  eine fachliche und überfachliche Studienberatung.
\end{itemize}

Die Belange von Studierenden mit Behinderung werden berücksichtigt.

\section{Leitfragen}\label{leitfragen}

\begin{itemize}
\item
  Woran erkennen die Verantwortlichen, dass die (formalen und
  fachlich-inhaltlichen) Zugangskriterien das Erreichen des angestrebten
  Kompetenzprofils unterstützen?
\item
  Ggf.: Wie wurde reagiert, wenn die Zugangsregelungen diesen Zweck aus
  Sicht der für den Studiengang Verantwortlichen nicht erfüllt haben?
\item
  Wie schätzen die für den Studiengang Verantwortlichen und daran
  Beteiligten~-- einschließlich der Studierenden -- die studentische
  Arbeitsbelastung ein? Welche Probleme treten auf? Was wird zu deren
  Lösung unternommen?
\item
  Sind hinsichtlich des Studienabschlusses in der vorgesehenen Zeit in
  den vergangenen Jahren Probleme aufgetreten? Wenn ja, welche? Wie
  wurden sie behandelt?
\item
  Inwieweit sind individuelle Mobilitätsfenster für Studierende im
  Studienverlauf realisierbar? Welche Probleme gibt es? Wie wurde darauf
  reagiert?
\item
  Welche Auswirkungen auf die Studierbarkeit haben die vorhandenen
  (prüfungsrelevanten) Regelungen zu Wiederholungsmöglichkeiten,
  Nachteilsausgleich bei Behinderung, Nichterscheinen im Krankheitsfall
  etc.?
\item
  Gab es Fälle, in denen sich die konkrete Prüfungsorganisation (z.~B.
  Terminierung der Prüfungen, Korrekturzeiten) nachteilig auf den
  Studienverlauf ausgewirkt haben? Wenn ja, welche Konsequenzen wurden
  gezogen?
\item
  Welche der vorhandenen Betreuungs- und Beratungsangebote für
  Studierende halten die für den Studiengang Verantwortlichen und
  Beteiligten -- einschließlich der Studierenden -- für besonders
  effektiv im Hinblick auf den Studienerfolg?
\item
  Welche Betreuungs- und Beratungsangebote für Studierende vermissen die
  für den Studiengang Verantwortlichen und Beteiligten -- einschließlich
  der Studierenden? Warum werden sie nicht realisiert?
\item
  Inwieweit werden Belange von Studierenden mit Behinderung
  berücksichtigt?
\end{itemize}

\section{Mögliche Evidenzen}\label{muxf6gliche-evidenzen}

\begin{itemize}
\item
  Ggf. Zugangssatzung sowie Informationen über die
  Studiengangsvoraussetzungen auf Webseiten, in Studienführern etc.
\item
  Einschlägige Ergebnisse interner Erhebungen und Evaluationen -- ggf.
  Daten zur studentischen Arbeitslast
\item
  Studienverlaufsplan, aus der/dem Semesterlage, Umfang und studentische
  Arbeitslast der Module pro Semester hervorgehen (ggf. mit
  Veröffentlichungsort wie z.~B. Homepage, Studienführer, Studien- bzw.
  Prüfungsordnungen) bzw. Dokumente, in denen Studienverläufe und deren
  Organisation geregelt sind
\item
  Dokumente, aus denen die geltenden Regelungen zur
  (Auslands-)Mobilität, Praxisphasen und Anerkennung von an anderen
  Hochschulen / außerhalb der Hochschule erbrachten Leistungen erkennbar
  sind
\item
  Dokumente aus dem täglichen Gebrauch an der Hochschule, aus denen das
  vorhandene Beratungs- und Betreuungskonzept hervorgeht
\item
  (statistische) Daten zu Studienverläufen
\item
  Ggf. Daten zur (Auslands-)Mobilität von Studierenden und zu
  Praxiseinsätzen von Studierenden
\item
  Ggf. weitere einschlägige Ergebnisse interner Befragungen und
  Evaluationen (auch Auffälligkeiten hinsichtlich der Wirkung von ggf.
  vorhandenen Maßnahmen zur Vermeidung von Ungleichbehandlungen in der
  Hochschule)
\end{itemize}

%

\chapter{Studiengangsbezogene Kooperationen}\label{Studiengangsbezogene Kooperationen}


Beteiligt oder beauftragt die Hochschule andere Organisationen mit der
Durchführung von Teilen des Studiengangs, gewährleistet sie die
Umsetzung und die Qualität des Studiengangskonzeptes. Umfang und Art
bestehender Kooperationen mit anderen Hochschulen, Unternehmen und
sonstigen Einrichtungen sind beschrieben und die der Kooperation zu
Grunde liegenden Vereinbarungen dokumentiert.

\section{Leitfragen}\label{leitfragen}

\begin{itemize}
\tightlist
\item
  Funktionieren die hochschulinternen und hochschulexternen
  Kooperationen aus Sicht der für den Studiengang Verantwortlichen?
\end{itemize}

\section{Mögliche Evidenzen}\label{muxf6gliche-evidenzen}

\begin{itemize}
\tightlist
\item
  Kooperationsverträge, Regeln für interne/externe Kooperationen
\end{itemize}

%



\chapter{Ausstattung}\label{Ausstattung}


Die adäquate Durchführung des Studiengangs ist hinsichtlich der
qualitativen und quantitativen personellen, sächlichen und räumlichen
Ausstattung gesichert. Dabei werden Verflechtungen mit anderen
Studiengängen berücksichtigt. Maßnahmen zur Personalentwicklung und
-qualifizierung sind vorhanden.

\section{Leitfragen}\label{leitfragen}

\begin{itemize}
\item
  Auf welche Weise stellen die für den Studiengang Verantwortlichen
  fest, dass Umfang und fachliche Qualifikation des Lehrpersonals für
  Lehre und Betreuung ausreichen?
\item
  Wie zufrieden sind die am Studiengang Beteiligten mit den Ressourcen
  für Lehre, Betreuung und Administration?
\item
  Wie reagieren die für den Studiengang Verantwortlichen auf auftretende
  Probleme und Engpässe?
\item
  Woran wird die Qualität von ggf. eingesetzten Lehrbeauftragten fest
  gemacht?
\item
  Inwieweit sind Forschungs- und Entwicklungstätigkeiten der Lehrenden
  der Studiengangsentwicklung förderlich?
\item
  Wer ist für die fachliche und didaktische Weiterentwicklung der
  Lehrenden verantwortlich?
\item
  Woran erkennen die Verantwortlichen, dass Weiterbildungsmaßnahmen
  erwünscht oder erforderlich sind?
\item
  Wie zufrieden sind die am Studiengang Beteiligten mit der sächlichen
  Ausstattung?
\item
  Wie reagieren die für den Studiengang Verantwortlichen auf Engpässe in
  der Ausstattung?
\end{itemize}

\section{Mögliche Evidenzen}\label{muxf6gliche-evidenzen}

\begin{itemize}
\item
  Beschreibung des Personals
\item
  Dokument aus dem täglichen Gebrauch der Hochschule, aus dem die
  ausreichende Lehrkapazität hervorgeht
\item
  Anzahl der Studierenden
\item
  Darstellung des didaktischen Weiterbildungsangebotes (ggf. Verweis auf
  Webseite) und von Maßnahmen zur Unterstützung der Lehrenden bei dessen
  Inanspruchnahme
\item
  Daten zu wahrgenommenen Weiterbildungsaktivitäten, z.~B.
  Forschungssemester, Gastprofessuren, Seminare, Tagungen, Workshops
\item
  (Kurz-)Darstellung der studiengangsbezogenen Forschungsaktivitäten
\item
  Dokumente aus dem täglichen Gebrauch der Hochschule, in denen die
  Ausstattung dargestellt wird, z.B. Laborhandbücher, Inventarlisten,
  Finanzpläne
\end{itemize}

%

\chapter{Transparenz und Dokumentation}\label{Transparenz und Dokumentation}


Studiengang, Studienverlauf, Prüfungsanforderungen und
Zugangsvoraussetzungen ein-schließlich der Nachteilsausgleichsregelungen
für Studierende mit Behinderung sind dokumentiert und veröffentlicht.

\section{Leitfragen}\label{leitfragen}

\begin{itemize}
\item
  Wie wird sichergestellt, dass inländische und ausländische Studierende
  ihre Rechte und Pflichten kennen?
\item
  Wer hat die Entscheidungsbefugnis über welche Dokumente?
\end{itemize}

\section{Mögliche Evidenzen}\label{muxf6gliche-evidenzen}

\begin{itemize}
\item
  Vorlage aller relevanten Regelungen zu Studienverlauf, Zugang,
  Studienabschluss, Prüfungen, Qualitätssicherung etc., mit Angabe zum
  Status der Verbindlichkeit
\item
  Verweis auf die Stelle, an der diese veröffentlicht sind, z.B.
  Webseiten
\end{itemize}

%

\chapter{Qualitätssicherung und Weiterentwicklung}\label{Qualitätssicherung und Weiterentwicklung}


Ergebnisse des hochschulinternen Qualitätsmanagements werden bei den
Weiterentwicklungen des Studienganges berücksichtigt. Dabei
berücksichtigt die Hochschule Evaluationsergebnisse, Untersuchungen der
studentischen Arbeitsbelastung, des Studienerfolgs und des
Absolventenverbleibs.

\section{Leitfragen}\label{leitfragen}

\begin{itemize}
\item
  Welche Maßnahmen zur Qualitätsverbesserung in und von Studiengängen
  sind in den zurückliegenden Jahren ergriffen worden?
\item
  Welche Elemente der internen Qualitätskontrolle erweisen sich als
  besonders nützlich für kontinuierliche Verbesserungen in einem
  Studiengang?
\item
  Inwieweit findet der Aspekt „Lernergebnisorientierung`` bei der
  Konzeption und in der Praxis der Qualitätssicherungsinstrumente für
  einen Studiengang Berücksichtigung?
\item
  Wie bewerten Studierende die interne Qualitätskontrolle
  und~-entwicklung ihrer Studiengänge hinsichtlich

  \begin{itemize}
  \item
    ihrer Beteiligung?
  \item
    der Auswirkungen auf ihr Studium?
  \end{itemize}
\item
  Wie bewerten Lehrende und die Leitungsebenen die interne
  Qualitätskontrolle und -entwicklung ihrer Studiengänge hinsichtlich

  \begin{itemize}
  \item
    ihrer Beteiligung?
  \item
    der Unterstützung bei der Lösung von Problemen und Verbesserungen in
    der Lehre?
  \end{itemize}
\end{itemize}

\section{Mögliche Evidenzen}\label{muxf6gliche-evidenzen}

\begin{itemize}
\item
  Interne Regelwerke zum Qualitätsmanagement (Evaluationsordnungen u.ä.)
\item
  Exemplarisches Informationsmaterial über das Qualitätsmanagement und
  seine Ergebnisse, das die Hochschule regelmäßig für die Kommunikation
  nach innen und außen nutzt (z.~B. Link zu spezifischen Webseiten,
  Berichte, Flyer)
\item
  Quantitative und qualitative Daten aus Befragungen, Statistiken zum
  Studienverlauf, Absolventenzahlen und -verbleib u.ä.
\end{itemize}

%



\chapter{Geschlechtergerechtigkeit und Chancengleichheit}\label{Geschlechtergerechtigkeit und Chancengleichheit}


Auf der Ebene des Studiengangs werden die Konzepte der Hochschule zur
Geschlechtergerechtigkeit und zur Förderung der Chancengleichheit von
Studierenden in besonderen Lebenslagen wie beispielsweise Studierende
mit gesundheitlichen Beeinträchtigungen, Studierende mit Kindern,
ausländische Studierende, Studierende mit Migrationshintergrund und/oder
aus sogenannten bildungsfernen Schichten umgesetzt.

\section{Leitfragen}\label{leitfragen}

\begin{itemize}
\tightlist
\item
  Liegen Konzepte der Hochschule zur Geschlechtergerechtigkeit und zur
  Förderung der Chancengleichheit von Studierenden in besonderen
  Lebenslagen vor? Wenn ja welche?
\end{itemize}

\section{Mögliche Evidenzen}\label{muxf6gliche-evidenzen}

\begin{itemize}
\tightlist
\item
  Einschlägige Dokumente aus dem alltäglichen Gebrauch der Hochschule,
  die die ggf. vorhandenen Konzepte und Maßnahmen zeigen
\end{itemize}

%

