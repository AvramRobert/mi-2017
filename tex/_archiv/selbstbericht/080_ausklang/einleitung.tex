%!TEX root = ../PP.tex

%TODO: 

\chapter{Fazit}

In der Bachelorarbeit wurde ein Leitfaden konzipiert, der für jeden Interessierten frei zugänglich ist. Der Leitfaden ist zu jeder Zeit editierbar und erweiterbar. Die Kombination aus \gls{videoclip}s und zusätzlichen Erklärungstexten auf der Webseite, gibt dem Leitfaden Raum um die Möglichkeiten und Grenzen des Klangbuchs zu veranschaulichen. Die Videoclips zeigen das Klangbuch, seine Funktionen und die Interaktionen aus Sicht eines Nutzers / Fans. So wird dem Besucher der Webseite nicht nur eine abstrakte Idee des digitalen Klangbuchs vermittelt. Der Besucher kann sich durch die Aufnahmeart der Videos in den Fan und dessen Erlebnis hinein versetzen.\\Die Möglichkeiten und Grenzen des digitalen Klangbuchs lassen sich mit dem konzipierten Leitfaden vermitteln.\\

Durch den Bau des Klangbuch-Prototypen konnte der Leitfaden ein real anmutendes digitales Klangbuch präsentieren. Mit dem Leitfaden ist es nun möglich das Interesse der Zielgruppe zu eruieren.\\

\textbf{Schwierigkeiten}\\
Im Rahmen der Bachelorarbeit konnte nicht ermittelt werden, wie verständlich der Leitfaden für die Zielgruppe tatsächlich ist und ob damit Interesse und Inspiration geweckt werden kann. Dies kann im Rahmen eines weiteren Projekts geprüft werden.\\

Da die \gls{videoclip}s bei natürlichem Licht gedreht wurden, sind weitere Funktionen nur schwer mit den gleichen Lichtverhältnissen zu filmen. Das Aufnahme-Setup sollte um einen Polfilter erweitert und mit Kunstlicht getestet werden.\\

Die Umsetzung des Leitfadens beschränkt sich vor allem auf die interaktiven und akustischen Funktionsmöglichkeiten des digitalen Klangbuchs. Das Integrieren der SMD-LEDs konnte nicht anschaulich integriert werden. In den Layouts der \gls{videoclip}s wurden sie aus zeitlichen Gründen nicht integriert.


%%%%%%%%%%%%%%%%%%%%%%%%%%%%%%%%%%%%%%%%%%%%%%%%%%%%%%
\chapter{Ausblick}
Mögliche weitere Schritte, um das Gesamtprojekt "Digitales Klangbuch" voran zu bringen, werden in diesem Kapitel kurz erläutert.\\


\textbf{Proof of Concept des Leitfadens}\\
Die Verständlichkeit des entwickelten Leitfadens kann anhand von Testpersonen, die bisher noch nicht mit dem Projekt "Digitales Klangbuch" vertraut sind, ermittelt werden.\\


\textbf{Weiterentwicklung Leitfaden}\\
Im Rahmen der Bachelorarbiet wurde ein reduzierter Leitfaden entwickelt, der nur einen kleinen Funktionsumfang des digitalen Klangbuchs erläutert. Der Leitfaden kann um weitere Funktionen und Inhalte erweitert werden. Der Leifaden als Webseite kann zudem um Animationen erweitert werden, die dem Besucher Interaktionen mit dem digitalen Klangbuch, in animierter, visualisierter Form ermöglichen.\\


\textbf{Zusammenarbeit mit KlangKünstlern}\\
Durch die Zusammenarbeit mit KlangKünstlern kann das digitale Klangbuch und auch der Leitfaden auf die Bedürfnisse des KlangKünstlers angepasst werden. Dadurch entsteht echtes Material, das direkt mit dem KlangKünstler diskutiert und an Fans getestet werden kann.\\


\textbf{Funktionaler Prototyp}\\
Ein funktionierender Prototyp, der eine ausgewählte Technologie, die Komponenten und die Programmierung enthält, wird entwickelt.\\


\textbf{Finanzierungsmöglichkeiten}\\
Die Entwicklung des digitalen Klangbuchs erfordert eine Finazierung. Diese könnte z. B. über einer \gls{crowdfunding}-Kampagne gedeckt werden.\\



%%%%%%%%%%%%%%%%%%%%%%%%%%%%%%%%%%%%%%%%%%%%%%%%%%%%%%
\chapter{Offene Fragen}

Folgende Fragen konnten in dieser Arbeit nicht beantwortet werden:\\

Wie verständlich ist der Leitfaden für Menschen, die nicht mit dem Projekt "Digitales Klangbuch" vertraut sind?\\
Kann das Interesse des KlangKünstlers an das digitale Klangbuch mit dem Leitfaden geweckt werden?\\
Kann der KlangKünstler mit dem Leitfaden inspiriert werden?\\
Welche Funktionsmöglichkeiten sind von KlangKünstlern gewünscht?\\












