%!TEX root = ../PP.tex

\chapter*{Kurzfassung}

Der Umgang mit und die Wertschätzung von Musik hat sich seit der Schallplatte bis zu den Streamingdiensten der heutigen Zeit stark verändert. Der Gesamtumsatz der Musik hat, seit Einführung der MP3, abgenommen, während der Konsum von Musik angestiegen ist.\cite{Musik2014} Musik ist heutzutage ein allgegenwärtiges, leicht zugängliches Konsumgut. Die abnehmende Wertschätzung ist ein Trend, der von Musikern, Künstlern und Musikfans gleichermaßen als Verlust bewertet wird.\cite{K01}\cite{Lost}\\

Eine Idee, um diesem Trend etwas entgegen zu setzen, ist das Schaffen eines neuen Mediums - das digitale Klangbuch. Das Konzept des digitalen Klangbuchs sieht ein großformatiges, physisches Buch als erweitertes, interaktives \gls{artwork} und sinnliches Gesamtpaket vor. Es soll Musik, Bilder, Grafiken und Informationen zum Musiker / Künstler und zur Musik enthalten und den Hörer / Fan interaktiv einbinden, fesseln und begeistern. Mit dem digitalen Klangbuch sollen mehrere Sinne gleichzeitig angesprochen werden: Der akustische, visuelle und der haptische Sinn.\\

Die Entwicklung des digitalen Klangbuchs ist ein Projekt, das aus vielen Teilprojekten besteht. Eines dieser Teilprojekte wurde im vorangegangen Praxisprojekt mit dem Thema „Konzeption eines erweiterten Buches als Tonträger zur Verknüpfung von Audio und Interaktion zur Aufwertung kreativer Audiowerke“ erarbeitet. Ziel des Praxisprojekt war die Beantwortung der Frage: Was sind geeignete technologische Ansätze um ein digitales Klangbuch zu entwickeln?\\

Die Bachelorarbeit setzt das vorangegangene Praxisprojekt fort.\\
Das digitale Klangbuch und dessen Möglichkeiten sind für außenstehende Personen nicht einfach zu verstehen. Ziel der Bachelorarbeit ist die Beantwortung der folgenden Frage: Wie lassen sich die Möglichkeiten und Grenzen des digitalen Klangbuchs gegenüber Musikern und Künstlern vermitteln? Zur Beantwortung dieser Frage wurden im ersten Schritt die Möglichkeiten und Grenzen des digitalen Klangbuchs erläutert. Daraufhin wurden Darstellungs- und Funktionsmöglichkeiten des Klangbuchs erarbeitet. Im Anschluss wurden Arten von Leitfäden recherchiert. Im nächsten Schritt wurde ein Leitfaden konzipiert und entwickelt: Der Aufbau des Leitfadens wurde diskutiert, Layouts gestaltet, Audiomaterial produziert und \gls{videoclip}s konzipiert, gedreht, montiert und vertont. Im Anschluss wurde ein Videokanal auf YouTube erstellt und eine Webseite entwickelt.\\

Die Arbeit wurde mit einem Fazit und einem Ausblick abgeschlossen.

