\chapter{Algorithmen und Programmierung
1}\label{algorithmen-und-programmierung-1}

\begin{modulHead}
\textbf{Modulverantwortlich}: Prof.~Dr.~Frank
Victor
\end{modulHead}
\begin{modulHead}
\textbf{Kürzel}:
AP1
\end{modulHead}
\begin{modulHead}
\textbf{Studiensemester}:
1
\end{modulHead}
\begin{modulHead}
\textbf{Sprache}:
deutsch
\end{modulHead}
\begin{modulHead}
\textbf{Kreditpunkte}:
8
\end{modulHead}
\begin{modulHead}
\textbf{Voraussetzungen nach
Prüfungsordnung}: keine
\end{modulHead}
\begin{modulHead}
\textbf{Typ}:
Pflichtmodul
\end{modulHead}


\section*{Lehrform/SWS:}\label{lehrformsws}

6 SWS: Vorlesung 3 SWS; Übung 1 SWS; Praktikum 2 SWS

\section*{Arbeitsaufwand:}\label{arbeitsaufwand}

Gesamtaufwand 240 h, davon

\begin{itemize}
\tightlist
\item
  54 h Vorlesung
\item
  36 h Praktikum
\item
  18 h Übung
\item
  132 h Selbststudium
\end{itemize}

\section*{Angestrebte
Lernergebnisse:}\label{angestrebte-lernergebnisse}

Die Studierenden sollen

\begin{itemize}
\tightlist
\item
  formale und algorithmische Kompetenzen im Bereich der
  Software-Entwicklung erlangen. Hierzu gehören insbesondere die
  Prinzipien der Objektorientierung und die der prozeduralen
  Programmierung.
\item
  die Kompetenz erlangen, strukturierte und unstrukturierte
  Problemstellungen zu analysieren, Lösungen modellbasiert zu entwickeln
  sowie prozedural und objektorientiert umzusetzen.
\item
  Systementwürfe evaluieren und bewerten können, insbesondere sollen sie
  die Arbeitsweise, die Randbedingungen und den Komplexitätsgrad von
  einfachen Algorithmen verstehen.
\item
  die Fähigkeit erlernen, algorithmische Entwurfsmuster zu erkennen und
  anzuwenden
\end{itemize}

\section*{Inhalt:}\label{inhalt}

\begin{itemize}
\tightlist
\item
  Prozedurale Programmierung am Beispiel von C.
\item
  Objektorientierte Programmierung am Beispiel von Java.
\item
  Kontroll- und Datenstrukturen.
\item
  Modularisierungskonzepte.
\item
  Typkonzepte.
\item
  Grundmuster der objektorientierten Programmierung.
\item
  Elementare Algorithmen und Aufwandsschätzung.
\item
  Entwicklungsumgebungen.
\end{itemize}

\section*{Studien-/Prüfungsleistungen:}\label{studien-pruxfcfungsleistungen}

Klausur sowie erfolgreiche Teilnahme am Praktikum als
Prüfungsvorleistung

\section*{Medienformen:}\label{medienformen}

\begin{itemize}
\tightlist
\item
  Beamer-gestützte Vorlesungen (Folien in elektronischer Form)
\item
  Praktikum an Rechnern des Labors
\end{itemize}

\section*{Literatur:}\label{literatur}

\begin{itemize}
\tightlist
\item
  Vorlesungsunterlagen: Foliensammlung, ausformuliertes Skript,
  Beispiellösungen, Übungsklausuren mit Lösungen
\item
  Fachliteratur: Diverse C-Bücher, u.a.: Kernighan, B.W., Ritchie, D.M.:
  „Programmieren in C``
\item
  Diverse Java-Bücher, u.a.: Bishop, J.: „Java Lernen``
\item
  Sedgewick, R.: „Algorithmen in Java``
\end{itemize}

\chapter{Algorithmen und Programmierung
2}\label{algorithmen-und-programmierung-2}

\begin{modulHead}
\textbf{Modulverantwortlich}: Prof.~Dr.~Christian
Kohls
\end{modulHead}
\begin{modulHead}
\textbf{Kürzel}:
AP2
\end{modulHead}
\begin{modulHead}
\textbf{Studiensemester}:
2
\end{modulHead}
\begin{modulHead}
\textbf{Sprache}:
deutsch
\end{modulHead}
\begin{modulHead}
\textbf{Kreditpunkte}:
7
\end{modulHead}
\begin{modulHead}
\textbf{Voraussetzungen nach
Prüfungsordnung}: Keine über die Zulassungsvorrausetzungen
hinausgehenden
Vorraussetzungen
\end{modulHead}
\begin{modulHead}
\textbf{Typ}:
Pflichtmodul
\end{modulHead}


\section*{Lehrform/SWS:}\label{lehrformsws-1}

6 SWS: Vorlesung 3 SWS; Übung 1 SWS; Praktikum 2 SWS

\section*{Arbeitsaufwand:}\label{arbeitsaufwand-1}

Gesamtaufwand 210 h, davon

\begin{itemize}
\tightlist
\item
  54 h Vorlesung
\item
  36 h Praktikum
\item
  18 h Übung
\item
  102 h Selbststudium
\end{itemize}

\section*{Angestrebte
Lernergebnisse:}\label{angestrebte-lernergebnisse-1}

Die Studierende sollen Objektorientierung, die Prinzipien der
Algorithmenentwicklung und grundlegende Algorithmen verstehen und die
Grundstrukturen der Java-Bibliothek anwenden können.

\section*{Inhalt:}\label{inhalt-1}

\begin{itemize}
\tightlist
\item
  Basisalgorithmen: Suchen u. Sortieren
\item
  Datenstrukturen
\item
  Dictionaries
\item
  Methodik des objektorientierten Programmierens
\end{itemize}

\section*{Studien-/Prüfungsleistungen:}\label{studien-pruxfcfungsleistungen-1}

Klausur sowie erfolgreiche Teilnahme am Praktikum als
Prüfungsvorleistung

\section*{Medienformen:}\label{medienformen-1}

\begin{itemize}
\tightlist
\item
  Beamer-gestützte Vorlesungen (Folien in elektronischer Form)
\item
  Praktikum an Rechnern des Labors
\end{itemize}

\section*{Literatur:}\label{literatur-1}

\begin{itemize}
\tightlist
\item
  Vorlesungsunterlagen: Foliensammlung, ausformuliertes Skript,
  Beispiellösungen
\item
  Fachliteratur: Bishop, J.: „Java Lernen``
\item
  Sedgewick, R.: „Algorithmen in Java``
\item
  Barnes, J., Kölling, M.: „Java Lernen mit BlueJ``, Verweise auf
  Onlinedokumente
\end{itemize}

\chapter{Audiovisuelles
Medienprojekt}\label{audiovisuelles-medienprojekt}

\begin{modulHead}
\textbf{Modulverantwortlich}: Prof.~Hans
Kornacher
\end{modulHead}
\begin{modulHead}
\textbf{Kürzel}:
AVM
\end{modulHead}
\begin{modulHead}
\textbf{Studiensemester}:
3
\end{modulHead}
\begin{modulHead}
\textbf{Sprache}:
deutsch
\end{modulHead}
\begin{modulHead}
\textbf{Kreditpunkte}:
5
\end{modulHead}
\begin{modulHead}
\textbf{Voraussetzungen nach
Prüfungsordnung}: Keine über die Zulassungsbedingungen hinausgehenden
Voraussetzungen
\end{modulHead}
\begin{modulHead}
\textbf{Typ}:
Pflichtmodul
\end{modulHead}


\section*{Lehrform/SWS:}\label{lehrformsws-2}

4 SWS: Vorlesung 2 SWS; Projekt 2 SWS

\section*{Arbeitsaufwand:}\label{arbeitsaufwand-2}

Gesamtaufwand 150 h, davon

\begin{itemize}
\tightlist
\item
  36 h Vorlesung
\item
  36 h Projektarbeit
\item
  78 h Selbststudium
\end{itemize}

\section*{Angestrebte
Lernergebnisse:}\label{angestrebte-lernergebnisse-2}

Die praktische Umsetzung des Vorlesungsstoffes, die Kommunikation und
Zusammenarbeit im Team über Themenbereiche dieses Faches und die
Präsentation von eigenen Projekten und Untersuchungsergebnissen sind die
Lernziele des Moduls „Audiovisuelles Medienprojekt``. Neben dieser
formulierten Fachkompetenz, Methodenkompetenz und
Kommunikationskompetenz stehen gerade die sogenannten Softskills
Teamfähigkeit und Kommunikationsfähigkeit im Focus der Ausbildung in
diesem Modul.

Um mit der sich hieraus entwickelnden erhöhten Komplexität der
Fragestellungen kompetent umgehen zu können, wird in diesem Modul
inhaltlich auf die technischen Grundlagen der audiovisuellen Produktion
besonders eingegangen. Die thematische Gewichtung der Inhalte ist im
Hinblick auf die Vermittlung der Befähigung ausgerichtet, selbstständig
in der audiovisuellen Produktion auftretende Problemstellungen lösen zu
können und die verwendeten technischen Werkzeuge, wie Videokamera,
Tonaufnahmegeräte und Schnittsysteme technisch richtig und gestalterisch
aussagekräftig einzusetzen. Über diese Methodenkompetenz hinaus wird in
Filmanalysen und einer kritischen Betrachtung medialer Ereignisse die
Fähigkeit zur Reflexion vermittelt. Dabei spielt die Begründung der
Auswahl bestimmter Medientechnologien und deren Einsatz in der
Medienproduktion eine wichtige Rolle.

Die Studierenden kennen die grundlegenden Erzählformen audiovisueller
Medien und haben folgende Fertigkeiten: Sie können eigene audiovisuelle
Erzählformen auf der Basis klassischer Erzählmuster entwickeln und sind
befähigt zur Analyse, Diskussion und zur kritischen Betrachtung
audiovisueller Medieninhalte.

Pragmatisches Ziel ist es, in den unterschiedlichsten Berufsfeldern
digitaler audiovisueller Medien die Entwicklung und den Einsatz
audiovisuellen Content zu beraten, zu planen, durchzuführen oder zu
verantworten.

\section*{Inhalt:}\label{inhalt-2}

Im Mittelpunkt dieses Moduls steht die digitale audiovisuelle
Medienproduktion.

Die Projektarbeit gliedert sich dabei in die selbstständige Entwicklung,
Ausarbeitung und Präsentation eines Filmthemas, in die praktische
Umsetzung in einem Filmprojekt und in die Nachbearbeitung und Montage in
einer dramaturgischen Erzählform.

Begleitend zu der Produktion werden folgende fachspezifischen Inhalte
thematisiert und in Übungsaufgaben vertieft:

\begin{itemize}
\tightlist
\item
  Video- und Audioaufnahmetechnik
\item
  Filmsprache
\item
  Lichtsetzung
\item
  Tonaufnahme
\item
  Dokumentarfilm und Interview
\item
  Dramaturgie
\item
  Schnitt und Montage
\end{itemize}

\section*{Studien-/Prüfungsleistungen:}\label{studien-pruxfcfungsleistungen-2}

Projektarbeit und schriftliche Ausarbeitung

\section*{Medienformen:}\label{medienformen-2}

\begin{itemize}
\tightlist
\item
  Beamer-gestützte Vorlesungen (Folien in elektronischer Form)
\item
  Beispiele aus verschiedenen Medien in elektronischer Form
\item
  Projektarbeit in Teams, um die erlernten Methoden und Techniken
  einzuüben und zu vertiefen;
\end{itemize}

\section*{Literatur:}\label{literatur-2}

\begin{itemize}
\tightlist
\item
  James Monaco, Film verstehen, Rowolth Taschenbuch Verlag Hamburg,
  1980, ISBN 3-499-162717
\item
  Syd Field, Drehbuchschreiben für Film und Fernsehen, München 2003,
  ISBN 354836473X
\item
  Steven D. Katz, Die Richtige Einstellung, Zweitausendeins, Frankfurt
  a.M.1998,ISBN 3-86150-229-1
\item
  David Lewis Yewdall, Practical Art of Motion Picture Sound, Focal
  Press, USA 2003, ISBN 0-240-80525-9
\item
  Hans Kornacher \& Manfred Stross, Dokumentarisches Videofilmen,
  Augustus Verlag, Augsburg, 1992, ISBN 3-8043-5474-2
\item
  Hans Beller Hg., Handbuch der Filmmontage, München: TR-Verlagsunion,
  1993, ISBN 3-8058-2357-6
\item
  Karel Reisz, Gavin Millar, Geschichte und Technik der Filmmontage,
  München: Filmlandpresse, 1988, ISBN 3-88690-071-1
\item
  Chris Vogler, Die Reise des Drehbuchschreibens, Verlag Zweitausendeins
\end{itemize}

\chapter{Bachelorarbeit}\label{bachelorarbeit}

\begin{modulHead}
\textbf{Modulverantwortlich}: alle Informatik
Professoren
\end{modulHead}
\begin{modulHead}
\textbf{Kürzel}:
BA
\end{modulHead}
\begin{modulHead}
\textbf{Studiensemester}:
6
\end{modulHead}
\begin{modulHead}
\textbf{Sprache}:
deutsch
\end{modulHead}
\begin{modulHead}
\textbf{Kreditpunkte}:
12
\end{modulHead}
\begin{modulHead}
\textbf{Voraussetzungen nach
Prüfungsordnung}: alle Modulprüfungen außer Bachelorarbeit und
Kolloquium bestanden
\end{modulHead}
\begin{modulHead}
\textbf{Typ}:
Pflichtmodul
\end{modulHead}


\section*{Lehrform/SWS:}\label{lehrformsws-3}

Angeleitetes, eigenverantwortliches Arbeiten

\section*{Arbeitsaufwand:}\label{arbeitsaufwand-3}

360 Stunden

\section*{Angestrebte
Lernergebnisse:}\label{angestrebte-lernergebnisse-3}

Die Bachelorarbeit soll zeigen, dass der Prüfling befähigt ist,
innerhalb einer vorgegebenen Frist eine praxisorientierte Aufgabe aus
seinem Fachgebiet sowohl in ihren fachlichen Einzelheiten als auch in
den fachübergreifenden Zusammenhängen nach wissenschaftlichen,
fachpraktischen und gestalterischen Methoden selbständig zu bearbeiten.
Die Bachelorarbeit ist in der Regel eine eigenständige Untersuchung mit
einer Aufgabenstellung aus der Medieninformatik und einer ausführlichen
Beschreibung und Erläuterung ihrer Lösung. In fachlich geeigneten Fällen
kann sie auch eine schriftliche Hausarbeit mit fachliterarischem Inhalt
sein.

\section*{Inhalt:}\label{inhalt-3}

Selbstständiges wissenschaftliches, fachpraktisches und gestalterisches
Bearbeiten einer Aufgabenstellung.

\chapter{Bachelor Kolloquium}\label{bachelor-kolloquium}

\begin{modulHead}
\textbf{Modulverantwortlich}: alle Informatik
Professoren
\end{modulHead}
\begin{modulHead}
\textbf{Kürzel}:
BK
\end{modulHead}
\begin{modulHead}
\textbf{Studiensemester}:
6
\end{modulHead}
\begin{modulHead}
\textbf{Sprache}:
deutsch
\end{modulHead}
\begin{modulHead}
\textbf{Kreditpunkte}:
3
\end{modulHead}
\begin{modulHead}
\textbf{Voraussetzungen nach
Prüfungsordnung}: alle Modulprüfungen außer Bachelor Kolloquium
bestanden
\end{modulHead}
\begin{modulHead}
\textbf{Typ}:
Pflichtmodul
\end{modulHead}


\section*{Lehrform/SWS:}\label{lehrformsws-4}

Angeleitetes, eigenverantwortliches Arbeiten

\section*{Arbeitsaufwand:}\label{arbeitsaufwand-4}

90 Stunden

\section*{Angestrebte
Lernergebnisse:}\label{angestrebte-lernergebnisse-4}

Das Kolloquium dient der Feststellung, ob der Prüfling befähigt ist, die
Ergebnisse der Bachelorarbeit, ihre fachlichen Grundlagen, ihre
fachübergreifenden Zusammenhänge und ihre außerfachlichen Bezüge
mündlich darzustellen und selbständig zu begründen und ihre Bedeutung
für die Praxis einzuschätzen. Dabei soll auch die Bearbeitung des Themas
der Bachelorarbeit mit dem Prüfling erörtert werden.

\section*{Inhalt:}\label{inhalt-4}

Vortrag über das Thema der Bachelorarbeit, Fachdiskussion und mündliche
Verteidigung der Arbeit im Kontext der Medieninformatik.

\section*{Studien-/Prüfungsleistungen:}\label{studien-pruxfcfungsleistungen-3}

mündliche Prüfung, Vortrag

\chapter{Betriebssysteme und verteilte
Systeme}\label{betriebssysteme-und-verteilte-systeme}

\begin{modulHead}
\textbf{Modulverantwortlich}: Prof.~Dr.~Matthias
Böhmer, Prof.~Dr.~Lutz
Köhler
\end{modulHead}
\begin{modulHead}
\textbf{Kürzel}:
BS1
\end{modulHead}
\begin{modulHead}
\textbf{Studiensemester}:
4
\end{modulHead}
\begin{modulHead}
\textbf{Sprache}:
deutsch
\end{modulHead}
\begin{modulHead}
\textbf{Kreditpunkte}:
5
\end{modulHead}
\begin{modulHead}
\textbf{Voraussetzungen nach
Prüfungsordnung}: keine
\end{modulHead}
\begin{modulHead}
\textbf{Typ}:
Pflichtmodul
\end{modulHead}


\section*{Kurzbeschreibung}\label{kurzbeschreibung}

Systemprogrammierung am Beispiel von UNIX.

\section*{Lehrform/SWS:}\label{lehrformsws-5}

4 SWS: Vorlesung 2 SWS; Praktikum 2 SWS

\section*{Arbeitsaufwand:}\label{arbeitsaufwand-5}

Gesamtaufwand 150 h, davon

\begin{itemize}
\tightlist
\item
  36 h Vorlesung
\item
  36 h Praktikum
\item
  78 h Selbststudium
\end{itemize}

\section*{Angestrebte
Lernergebnisse:}\label{angestrebte-lernergebnisse-5}

Die Studierenden sollen:

\begin{itemize}
\tightlist
\item
  die Prinzipien und Mechanismen von Betriebssystemen und verteilten
  Systemen am Beispiel von UNIX verstanden haben,
\item
  in der Lage sein, selbstständig Systemprogramme zu schreiben und
  Betriebssystemstrukturen zu bewerten und
\item
  die Mechanismen zur Implementierung verteilter Anwendungen anwenden
  können.
\end{itemize}

\section*{Inhalt:}\label{inhalt-5}

Systemprogrammierung am Beispiel von UNIX:

\begin{itemize}
\tightlist
\item
  Shell-Programmierung
\item
  Prozess-Modelle
\item
  Prozess-Erzeugung und Synchronisation
\item
  UNIX-Prozesse und elementare Synchronisation
\item
  Pipes
\item
  Shared Memory
\item
  Synchronisationsprimitive für den wechselseitigen Ausschluss
\item
  Semaphoren
\item
  Nachrichtenwarteschlangen
\item
  Dateisysteme
\item
  TCP/IP
\item
  Sockets
\item
  Remote Procedure Call
\item
  Strategien zum Scheduling und zur Speicherverwaltung
\item
  Klassische Synchronisationsprobleme
\end{itemize}

\section*{Studien-/Prüfungsleistungen:}\label{studien-pruxfcfungsleistungen-4}

Klausur sowie erfolgreiche Teilnahme am Praktikum als
Prüfungsvorleistung.

\section*{Medienformen:}\label{medienformen-3}

Foliensammlung, ausformuliertes Skript, Beispiellösungen

\section*{Literatur:}\label{literatur-3}

\begin{itemize}
\tightlist
\item
  Tanenbaum, A. S.: „Moderne Betriebssysteme``
\item
  Brown, C.: „Programmieren verteilter UNIX-Anwendungen``
\item
  Kernighan, B. W., Pike, R.: „Der UNIX-Werkzeugkasten``
\item
  Ehses, E., Köhler, L., Stenzel, H., Victor, F. „Betriebssysteme: Ein
  Lehrbuch mit Übungen zur Systemprogrammierung in UNIX/Linux``
\end{itemize}

\chapter{BWL I - Grundlagen}\label{bwl-i---grundlagen}

\begin{modulHead}
\textbf{Modulverantwortlich}: Prof.~Dr.~Monika
Engelen
\end{modulHead}
\begin{modulHead}
\textbf{Kürzel}:
BWL1
\end{modulHead}
\begin{modulHead}
\textbf{Studiensemester}:
5
\end{modulHead}
\begin{modulHead}
\textbf{Sprache}:
deutsch
\end{modulHead}
\begin{modulHead}
\textbf{Kreditpunkte}:
5
\end{modulHead}
\begin{modulHead}
\textbf{Voraussetzungen nach
Prüfungsordnung}: keine
\end{modulHead}
\begin{modulHead}
\textbf{Typ}:
Pflichtmodul
\end{modulHead}


\section*{Lehrform/SWS:}\label{lehrformsws-6}

4 SWS: Vorlesung 2 SWS; Übung 2 SWS

\section*{Arbeitsaufwand:}\label{arbeitsaufwand-6}

Gesamtaufwand 120 h, davon

\begin{itemize}
\tightlist
\item
  30 h Vorlesung
\item
  30 h Übung
\item
  60 h Selbstlernphase
\end{itemize}

\section*{Angestrebte
Lernergebnisse:}\label{angestrebte-lernergebnisse-6}

Die Studierenden verstehen die wichtigsten Entscheidungsbereiche
wirtschaftlichen Handeln und können diese anwenden. Sie können
grundlegenden Entscheidungen im Rahmen einer Unternehmensgründung, die
Aufgaben der Unternehmensführung wie die Konzeption einer tragfähigen
Strategie, und die Aufgaben der Teilbereiche Produktion, Absatz und
Marketing sowie Investition und Finanzierung beschreiben und beurteilen.
Investitionsentscheidungen können die Studierenden informationsgestützt
treffen indem Sie die Kalkulationsverfahren der Investitionsrechnung
anwenden und auswerten. Die Veranstaltung bereitet die Studierenden für
weitere BWL-Veranstaltungen Ihres Studiums, sowie darauf, in ihrem
Berufsleben wirtschaftliche Konzepte im Unternehmenskontext anzuwenden,
vor.

\section*{Inhalt:}\label{inhalt-6}

\begin{itemize}
\tightlist
\item
  Grundlagen
\item
  Unternehmensführung 1: Ziele, Planung und Entscheidung
\item
  Investition und Finanzierung
\item
  Unternehmensführung 2: Ausführung und Kontrolle
\item
  Konstitutive Entscheidungen
\item
  Produktion
\item
  Absatz und Marketing
\end{itemize}

\section*{Studien-/Prüfungsleistungen:}\label{studien-pruxfcfungsleistungen-5}

schriftliche Klausur

\section*{Literatur:}\label{literatur-4}

\begin{itemize}
\tightlist
\item
  Wöhe (2016): Einführung in die Allgemeine Betriebswirtschaftslehre,
  26. Aufl.
\end{itemize}

\chapter{Datenbanken 1}\label{datenbanken-1}

\begin{modulHead}
\textbf{Modulverantwortlich}: Prof.~Dr.~Birgit
Bertelsmeier, Prof.~Dr.~Heide
Faeskorn-Woyke
\end{modulHead}
\begin{modulHead}
\textbf{Kürzel}:
DB1
\end{modulHead}
\begin{modulHead}
\textbf{Studiensemester}:
3
\end{modulHead}
\begin{modulHead}
\textbf{Sprache}:
Deutsch
\end{modulHead}
\begin{modulHead}
\textbf{Kreditpunkte}:
5
\end{modulHead}
\begin{modulHead}
\textbf{Voraussetzungen nach
Prüfungsordnung}: Klausurteilnahme nur bei bestandenem
DBS1-Praktikum
\end{modulHead}
\begin{modulHead}
\textbf{Typ}:
Pflichtmodul
\end{modulHead}


\section*{Lehrform/SWS:}\label{lehrformsws-7}

5 SWS: Vorlesung 2 SWS; Übung 1 SWS; Praktikum 1 SWS

\section*{Arbeitsaufwand:}\label{arbeitsaufwand-7}

Gesamtaufwand 150 h, davon

\begin{itemize}
\tightlist
\item
  36 h Vorlesung
\item
  18 h Praktikum
\item
  18 h Übung
\item
  78 h Selbststudium
\end{itemize}

\section*{Angestrebte
Lernergebnisse:}\label{angestrebte-lernergebnisse-7}

Die Studierenden sollen

\begin{itemize}
\tightlist
\item
  über ein einheitliches konsistentes Begriffsgebäude bezüglich der
  Datenbankthematik verfügen,
\item
  die theoretischen Grundlagen von Datenbanksystemen am Beispiel
  relationaler und objektrelationaler Datenbanksysteme verstanden haben,
  insbesondere die relationale Algebra, die Normalisierung sowie
  funktionale Abhängigkeiten
\item
  in der Lage sein, diese Erkenntnisse im Rahmen der Modellierung und
  Implementierung von Datenbankschemata praktisch anzuwenden,
\item
  komplexere Datenbankanfragen, Datendefinitionen und Datenänderungen
  über SQL programmieren~ können
\item
  ein SQL-Statement tunen können
\item
  mit dem Transaktionsbegriff, der Mehrbenutzersynchronisation und
  Verfahren zur Fehlererholung sowie zur Sicherung der Datenintegrität
  vertraut sein
\end{itemize}

\section*{Inhalt:}\label{inhalt-7}

\begin{itemize}
\tightlist
\item
  Grundbegriffe und Architektur von Datenbanken
\item
  Ein Vorgehensmodell zur Erstellung eines Datenbanksystems
\item
  Grundlagen des relationalen Modells
\item
  Relationale Algebra
\item
  Anfrageoptimierung
\item
  Funktionale Abhängigkeiten
\item
  Datenintegrität
\item
  Normalisierung
\item
  Datenmodellierung (Entity Relationship Modell) und Implementierung am
  Beispiel eines relationalen Datenbanksystems
\item
  Datenbanksprache SQL: DDL, DML, DAL, Integritätsbedingungen und
  Constraints unter dem jeweils aktuellen SQL-Standard, zur Zeit SQL2013
\item
  Transaktionskonzepte, Mehrbenutzersynchronisation, Fehlererholung und
  Datensicherheit
\end{itemize}

\section*{Studien-/Prüfungsleistungen:}\label{studien-pruxfcfungsleistungen-6}

Klausur, deren Voraussetzung das bestandene Praktikum ist, und
semesterbegleitende Multiple-Choice-Tests mit Punkten für die Klausur

\section*{Medienformen:}\label{medienformen-4}

\begin{itemize}
\tightlist
\item
  Folien gestützer Vortrag - aber nur sehr selten
\item
  I.d.R. erarbeiten der Theorie anhand von überschaubaren
  Problemstellungen und deren in der Veranstaltung entwickelten Lösungen
  - hauptsächliches Vorgehen
\item
  Fragen der Studierenden beantworten - sehr erwünscht!
\item
  Ilias zur Bereitstellung aller Informationen (Aktuelles, Links,
  Folien, Praktikums-/Übungsaufgaben, wie auch Lösungen)
\item
  edb, die DB-eLearning-Plattform der TH Köln
\item
  DB-Wiki, das Online Lexikon für Datenbank-Themen
\end{itemize}

\section*{Literatur:}\label{literatur-5}

\begin{itemize}
\tightlist
\item
  Elmasri, R.; Navathe, S. B.: Grundlagen von Datenbanksystemen.
  Pearson-Studium. 2009
\item
  Faeskorn-Woyke, H.; Bertelsmeier, B.; Riemer, P.; Bauer, E.:
  Datenbanksysteme - Theorie und Praxis mit SQL2003, Oracle und MySQL.
  Pearson-Studium. 2. Aufl. 2011
\item
  Kemper, A.; Eickler, A.: Datenbanksysteme -- Eine Einführung.
  Oldenbourg-Verlag, 2015
\item
  Saake, G., Sattler, K.-U.; Heuer, A.: Datenbanken - Konzepte und
  Sprachen. Mitp/bhv, 2013
\item
  Vossen, G.: Datenmodelle, Datenbanksprachen,
  Datenbankmanagementsysteme. Oldenbourg-Verlag, 2008
\end{itemize}

\chapter{Einführung in Betriebssysteme und
Rechnerarchitektur}\label{einfuxfchrung-in-betriebssysteme-und-rechnerarchitektur}

\begin{modulHead}
\textbf{Modulverantwortlich}: Prof.~Dr.~Stefan
Karsch
\end{modulHead}
\begin{modulHead}
\textbf{Kürzel}:
EBR
\end{modulHead}
\begin{modulHead}
\textbf{Studiensemester}:
1
\end{modulHead}
\begin{modulHead}
\textbf{Sprache}:
deutsch
\end{modulHead}
\begin{modulHead}
\textbf{Kreditpunkte}:
5
\end{modulHead}
\begin{modulHead}
\textbf{Voraussetzungen nach
Prüfungsordnung}: Erfolgreiche Modulprüfungen in den Modulen des
Grundstudiums
\end{modulHead}
\begin{modulHead}
\textbf{Typ}:
Pflichtmodul
\end{modulHead}


\section*{Lehrform/SWS:}\label{lehrformsws-8}

4 SWS: Vorlesung 2 SWS; Übung 2 SWS

\section*{Arbeitsaufwand:}\label{arbeitsaufwand-8}

Gesamtaufwand 150 h, davon

\begin{itemize}
\tightlist
\item
  36 h Vorlesung
\item
  36 h Übung
\item
  78 h Selbststudium
\end{itemize}

\section*{Angestrebte
Lernergebnisse:}\label{angestrebte-lernergebnisse-8}

Die Studierenden sollen die Basiskonzepte und Grundlagen der
Betriebssysteme und der Rechnerarchitektur kennen und verstehen, sowie
ein einheitliches konsistentes Begriffsgebäude zu, teilweise aus der
persönlichen Praxis bekannten, Sachverhalten der IT aufbauen

\section*{Inhalt:}\label{inhalt-8}

\begin{itemize}
\tightlist
\item
  Betriebssysteme aus Nutzersicht: Dateisysteme, Parallele Prozesse,
  Sicherheit in Betriebssystemen
\item
  bei Rechnerkomponenten: grundlegende Architekturen, Darstellung von
  Daten, interne Bussysteme, Prozessoren, Festplatten,
  Peripherieschnittstellen, Parallelrechner
\end{itemize}

\section*{Studien-/Prüfungsleistungen:}\label{studien-pruxfcfungsleistungen-7}

Schriftliche Prüfung, zuvor erfolgreiche Teilnahme am Praktikum als
Prüfungsvorleistung

\section*{Literatur:}\label{literatur-6}

\begin{itemize}
\tightlist
\item
  Vorlesungsunterlagen: kommentierte Foliensammlung
\item
  Tanenbaum: „Rechnerarchitektur``
\item
  Tanenbaum: „Modern Operating Systems``
\end{itemize}

\chapter{Einführung in die
Medieninformatik}\label{einfuxfchrung-in-die-medieninformatik}

\begin{modulHead}
\textbf{Modulverantwortlich}: Prof.~Dr.~Martin
Eisemann, Prof.~Dr.~Kristian Fischer, Prof.~Dr.~Gerhard Hartmann,
Prof.~Dr.~Christian Kohls, Prof.~Hans Kornacher, Prof.~Christian Noss,
Prof.~Dr.~Mario
Winter
\end{modulHead}
\begin{modulHead}
\textbf{Kürzel}:
EMI
\end{modulHead}
\begin{modulHead}
\textbf{Studiensemester}:
1
\end{modulHead}
\begin{modulHead}
\textbf{Sprache}:
Deutsch
\end{modulHead}
\begin{modulHead}
\textbf{Kreditpunkte}:
5
\end{modulHead}
\begin{modulHead}
\textbf{Voraussetzungen nach
Prüfungsordnung}: Keine
\end{modulHead}
\begin{modulHead}
\textbf{Typ}:
Pflichtmodul
\end{modulHead}


\section*{Lehrform/SWS:}\label{lehrformsws-9}

Vorlesung mit eingebetteten Übungselementen

\section*{Arbeitsaufwand:}\label{arbeitsaufwand-9}

150 Stunden

\section*{Angestrebte
Lernergebnisse:}\label{angestrebte-lernergebnisse-9}

Die Studierenden können die inhaltlichen Ausrichtungen und die
Zielsetzungen der Lehr- und Anwendungsdisziplin Medieninformatik
benennen und gegenüber verwandten oder ähnlichen Disziplinen abgrenzen.

Die Studierenden kennen Grundkonzepte der Informatik (z.B.
Anforderungen) sowie audiovisueller und interaktiver Medientechnologien,
kennen architekturelle Alternativen interaktiver Systeme und kennen
Gestaltungsdimensionen für deren Informations- und
Kommunikationsinhalte. Die Studierenden sind sensibilisiert für
Modellierungs- und Entwicklungsaufgaben von medienbasierten
Software-Systemen zur Unterstützung menschlichen Handelns in
betrieblichen und privaten Kontexten.

Sie kennen grundlegende Konzepte, Prozesse/Verfahren und Modelle der
Medieninformatik und haben erste Projekterfahrungen gesammelt. Sie
können Systemkonzeptionen, zugehörige Modellierungen, Abwägungen und
Artefakte für ein Fachpublikum angemessen dokumentieren und mittels
verschiedener medialer Formen kommunizieren.

\section*{Inhalt:}\label{inhalt-9}

Workshops zu grundlegenden projektrelevanten Themenfeldern (wie:
Datenmodellierung, Pseudo-Code, Kommunikation in verteilen medialen
Systeme, Visual Thinking, Storytelling, Anforderungen) und deren
Anwendung, illustriert anhand von Fallstudien.

Teambasiertes Projekt, welches ausgehend von Kontextszenarien eine (oder
mehrere) Problemstellung(en) umreist, zu dem Lösungen konzipiert und
prototypisch umgesetzt, dokumentiert und einem Fachpublikum präsentiert
werden müssen.

\section*{Studien-/Prüfungsleistungen:}\label{studien-pruxfcfungsleistungen-8}

mündliche Prüfung, Projektpräsentation, schriftliche Ausarbeitung

\section*{Medienformen:}\label{medienformen-5}

\begin{itemize}
\tightlist
\item
  Beamer-gestützte Vorlesungen (Folien in elektronischer Form)
\item
  Vorträge
\item
  verschiedene Präsentationsmaterialien (Whiteboard, Poster, etc.)
\item
  Einsatz von Bild- und Videobearbeitungssoftware
\item
  Umgang mit Kameras im Projektteil
\end{itemize}

\section*{Literatur:}\label{literatur-7}

\begin{itemize}
\tightlist
\item
  Michael Herczeg: Einführung in die Medieninformatik, Oldenbourg
  Verlag, 2006, ISBN: 3-486-581-031
\item
  Chris Rupp et al: Requirements-Engineering und -Management: Aus der
  Praxis von klassisch bis agil, Carl Hanser Verlag; 6-te Auflage, 2014,
  ISBN-10: 3446438939
\end{itemize}

\chapter{Entwicklungsprojekt}\label{entwicklungsprojekt}

\begin{modulHead}
\textbf{Modulverantwortlich}: Prof.~Dr.~Martin
Eisemann, Prof.~Dr.~Kristian Fischer, Prof.~Dr.~Gerhard Hartmann,
Prof.~Dr.~Christian Kohls, Prof.~Hans Kornacher, Prof.~Christian Noss,
Prof.~Dr.~Mario
Winter
\end{modulHead}
\begin{modulHead}
\textbf{Kürzel}:
EPM
\end{modulHead}
\begin{modulHead}
\textbf{Studiensemester}:
5
\end{modulHead}
\begin{modulHead}
\textbf{Sprache}:
deutsch
\end{modulHead}
\begin{modulHead}
\textbf{Kreditpunkte}:
10
\end{modulHead}
\begin{modulHead}
\textbf{Voraussetzungen nach
Prüfungsordnung}:
Schwerpunktmodul
\end{modulHead}
\begin{modulHead}
\textbf{Typ}:
Pflichtmodul
\end{modulHead}


\section*{Lehrform/SWS:}\label{lehrformsws-10}

Projekt

\section*{Arbeitsaufwand:}\label{arbeitsaufwand-10}

300 h

\section*{Angestrebte
Lernergebnisse:}\label{angestrebte-lernergebnisse-10}

Die Studierenden sollen vertiefende Kenntnisse in die Methoden und
Techniken aus zwei ausgewählten Modulen aus den ersten vier
Fachsemestern des Studiums erlangen und diese in der Konzeption und
prototypischen Realisierung eines interaktiven Systems oder Mediums
anwenden. Dadurch sollen sie eigene Erfahrungen in der Projektabwicklung
mit Medieninformatik-spezifischen Fragestellungen und in der Teamarbeit
sammeln und eine reflektierend-kritische Haltung zu methodischen
Ansätzen und Entwicklungsmodellen entwickeln. Ziel ist es eine, mit
eigenen praktischen Erfahrungen fundierte Methodenkompetenz zu erlangen.

Die Studierenden sollen darüberhinaus lernen, die Vorgehensweise und die
Ergebnisse ihres Projektes in einem kritischen Diskurs vor einem
Fachpublikum zu vertreten, um in der Berufspraxis ihre Herangehensweise
und Projektergebnisse vertreten zu können.

\section*{Inhalt:}\label{inhalt-10}

Die Projekte werden in Teams durchgeführt. Zunächst werden von den Teams
zwei Module aus den ersten vier Fachsemestern gewählt, welche die
fachlichen Perspektiven für die Vertiefung bestimmen. In Absprache mit
den Lehrenden werden dann Projektziele festgelegt.

Aus dem Methoden- und Technikkatalog (siehe Vorlesungen zu den
Lehrbereichen) wird in Absprache mit den Lehrenden eine Auswahl der
einzusetzenden Entwicklungstechniken und -methoden sowie der
einzuhaltenden Entwicklungsmodelle getroffen und
Qualitätssicherungsmaßnahmen und das Prozessmanagement definiert.

Die Lehrenden bieten dann während der Projektdurchführung Beratung bzgl.
des adäquaten Einsatzes der gewählten Methoden und Techniken.
Zwischenstände des Projektes werden zu definierten Meilensteinen von den
Projektteams präsentiert. Die Präsentation der Projektergebnisse findet
in einem Plenum mit kritischer Diskussion der Methoden und Ergebnisse
statt.

\section*{Studien-/Prüfungsleistungen:}\label{studien-pruxfcfungsleistungen-9}

Die Projektergebnis, bestehend aus Prototyp und Dokumentation, geht mit
80\% in die Projektnote ein, die Kommunikation der Zwischenergebnisse
und des Endergebnisses mit 20\%.

\chapter{Grundlagen des Web}\label{grundlagen-des-web}

\begin{modulHead}
\textbf{Modulverantwortlich}: Prof.~Dr.~Kristian
Fischer
\end{modulHead}
\begin{modulHead}
\textbf{Kürzel}:
GW
\end{modulHead}
\begin{modulHead}
\textbf{Studiensemester}:
3
\end{modulHead}
\begin{modulHead}
\textbf{Sprache}:
deutsch
\end{modulHead}
\begin{modulHead}
\textbf{Kreditpunkte}:
5
\end{modulHead}
\begin{modulHead}
\textbf{Voraussetzungen nach
Prüfungsordnung}: keine
\end{modulHead}
\begin{modulHead}
\textbf{Typ}:
Pflichtmodul
\end{modulHead}


\section*{Kurzbeschreibung}\label{kurzbeschreibung-1}

In der Veranstaltung werden wesentliche Grundideen,
Interaktionsprinzipien, Contentarchitekturen und Sicherheitsmechanismen
eingeführt, die das Web als Medium konstituieren.

\section*{Lehrform/SWS:}\label{lehrformsws-11}

4 SWS: Vorlesung 2 SWS; Seminar 2 SWS

\section*{Arbeitsaufwand:}\label{arbeitsaufwand-11}

Gesamtaufwand 150 h, davon

\begin{itemize}
\tightlist
\item
  36 h Vorlesung
\item
  36 h Seminar
\item
  78 h Selbststudium
\end{itemize}

\section*{Angestrebte
Lernergebnisse:}\label{angestrebte-lernergebnisse-11}

Die Studierenden

\begin{itemize}
\tightlist
\item
  kennen wesentliche Grundideen, Interaktionsprinzipien,
  Contentarchitekturen und Sicherheitsmechanismen, die das Web als
  Medium konstituieren und
\item
  können moderne Webanwendungen auf der Basis von Fachbegriffen
  analysieren und einordnen
\item
  um kompetent am fachlichen Diskurs über Eigenschaften, Auswirkungen
  und Gestaltungsalternativen von Web Anwendungen teilnehmen zu können.
\end{itemize}

\section*{Inhalt:}\label{inhalt-11}

\begin{itemize}
\tightlist
\item
  Web Architektur des W3C
\item
  Offfenheit und Verwendung von Standards als Prinzip
\item
  Interaktionsformen: Synchrone Interaktion auf der Basis von REST,
  asynchrone Interaktion mit Publish/Subscribe
\item
  Fallstudien: Cloudservices für verteilte Anwendungen - z.B. Amazon Web
  Services, Google Firebase
\item
  Ausgewählte Sicherheitsmechanismen im Web
\item
  Inhaltsarchitekturen: XML, JSON, Microformate, RDFa
\end{itemize}

Die Inhalte werden als Vorlesung vermittelt. In dem begleitenden Seminar
werden die Konzepte mittels Fallstudien anwendungsbezogen analysiert und
diskutiert.

\section*{Studien-/Prüfungsleistungen:}\label{studien-pruxfcfungsleistungen-10}

Mündliche Prüfung

\section*{Medienformen:}\label{medienformen-6}

\begin{itemize}
\tightlist
\item
  Folienpräsentation
\item
  Auschnitte aus der Literatur als Leseaufgaben und Fallstudien
\end{itemize}

\section*{Literatur:}\label{literatur-8}

\begin{itemize}
\tightlist
\item
  Randy Conolly, Richard Hoar: Fundamentals of Web Development, Pearson
  Publishing 2015
\item
  Hugh Taylor et al.: Event-Driven Architecture - How SOA Enables the
  Real-Time Enterprise, Addison-Wesley 2009
\item
  Webber: REST in Practice, OReilly 2011
\item
  Sam Newman: Building Micro Services, OReilly 2015
\item
  James Governor et al.: Web 2.0 Architectures, OReilly 2009
\item
  Rajkumar Buyya (ed.): Internet of Things: Principles and Paradigms,
  Morgan Kaufmann 2016
\end{itemize}

\chapter{Kommunikationstechnik und
Netze}\label{kommunikationstechnik-und-netze}

\begin{modulHead}
\textbf{Modulverantwortlich}: Prof.~Dr.~Hans L.
Stahl
\end{modulHead}
\begin{modulHead}
\textbf{Kürzel}:
KTN
\end{modulHead}
\begin{modulHead}
\textbf{Studiensemester}:
3
\end{modulHead}
\begin{modulHead}
\textbf{Sprache}:
Deutsch
\end{modulHead}
\begin{modulHead}
\textbf{Kreditpunkte}:
5
\end{modulHead}
\begin{modulHead}
\textbf{Voraussetzungen nach
Prüfungsordnung}: Kenntnisse aus dem
Grundstudium.
\end{modulHead}
\begin{modulHead}
\textbf{Typ}:
Pflichtmodul
\end{modulHead}


\section*{Lehrform/SWS:}\label{lehrformsws-12}

Vorlesung, Praktikum

\section*{Angestrebte
Lernergebnisse:}\label{angestrebte-lernergebnisse-12}

Die Studierenden sollen

\begin{itemize}
\tightlist
\item
  Prinzipien und Grundlagen von technischen Kommunikations­vor­gängen
  kennen lernen,
\item
  Protokolle als wesentliche Grundlage der Kommunikationstechnik im
  Detail verstehen (Internet-Protokolle, Multimedia-Protokolle,
  TK-Protokolle, Dienste)
\item
  Einsatz und Nutzung von Kommunikations­tech­nik praxistypisch kennen
  lernen,
\item
  in der Lage sein, selbstständig Netzstrukturen zu bewerten, Netze zu
  analysieren und zu konzipieren (unter Anwendung von
  Netz­analyse­werkzeugen und -methoden).
\end{itemize}

\section*{Inhalt:}\label{inhalt-12}

Grundbegriffe und Grundlagen, Kommunikationssysteme (Modelle,
Grundbegriffe), Protokolle, Schnittstellen, Dienste, Architekturmodelle
(OSI-Referenzmodell, TCP/IP-Protokollfamilie), Standardisierung,
TCP/IP-Protokollfamilie als Grundlage des Internet, Schichtenmodell und
Protokolle im Detail, Adressierung, ausgewählte Anwendungen,
Klassifizierung von Netzen / Topologien / Technologien, Wegewahl /
Vermittlung / Routing, Vermittlungsprinzipien, Routing-Verfahren und~
Protokolle, Internet-spezifische Verfahren, Multimedia-Netze,
Dienstgüte, Internet-Telefonie, Realisierung von Multimedia-Netzen,
Netzsicherheit, grundlegende Begriffe der „IT-Sicherheit``, typische
Bedrohungen in Netzen, Beispielszenarien

\section*{Medienformen:}\label{medienformen-7}

\begin{itemize}
\tightlist
\item
  Vorlesung im Hörsaal (PowerPoint und Beamer)
\item
  Praktikum an Rechnern des KTDS-Labors; Ressourcen:
  Netzanalysesoftware,div. Netzüberwachungssoftware, E-Mail-Server und
  -Clients, DNS-Server, ggf. weitereServer-Implementierungen
\end{itemize}

\section*{Literatur:}\label{literatur-9}

\begin{itemize}
\tightlist
\item
  Wird in der Veranstaltung bekannt gegeben
\end{itemize}

\chapter{Mathematik 1}\label{mathematik-1}

\begin{modulHead}
\textbf{Modulverantwortlich}: Prof.~Dr.~Wolfgang
Konen
\end{modulHead}
\begin{modulHead}
\textbf{Kürzel}:
MA1
\end{modulHead}
\begin{modulHead}
\textbf{Studiensemester}:
1
\end{modulHead}
\begin{modulHead}
\textbf{Sprache}:
deutsch
\end{modulHead}
\begin{modulHead}
\textbf{Kreditpunkte}:
7
\end{modulHead}
\begin{modulHead}
\textbf{Voraussetzungen nach
Prüfungsordnung}: ~
\end{modulHead}
\begin{modulHead}
\textbf{Typ}:
Pflichtmodul
\end{modulHead}


\section*{Lehrform/SWS:}\label{lehrformsws-13}

6 SWS: Vorlesung 3 SWS; Praktikum 1 SWS; Übung 2 SWS

\section*{Arbeitsaufwand:}\label{arbeitsaufwand-12}

Gesamtaufwand 210 h, davon

\begin{itemize}
\tightlist
\item
  54 h Vorlesung
\item
  18 h Praktikum
\item
  36 h Übung
\item
  102 h Selbststudium
\end{itemize}

\section*{Angestrebte
Lernergebnisse:}\label{angestrebte-lernergebnisse-13}

Die Studierenden sollen die Fähigkeiten zur Analyse realer oder
geplanter Systeme entwickeln, indem sie praktische Aufgabenstellungen
aus dem Informatik-Umfeld in mathematische Strukturen abstrahieren und
lernen, selbstständig die Modellfindung und die Ergebnisbeurteilung
vorzunehmen. Dabei sollen die Anwendungsbezüge der Mathematik deutlich
werden, z.B. die Bedeutung funktionaler Beziehungen für kontinuierliche
Zusammenhänge, die lineare Algebra z.B als Grundlage der grafischen
Datenverarbeitung und die Analysis zur Verarbeitung von Signalen und zur
Lösung von mathematischen Modellen.

\section*{Inhalt:}\label{inhalt-13}

\begin{itemize}
\tightlist
\item
  Grundlagen
\item
  Folgen
\item
  Funktionen
\item
  Differenzialrechnung (1 Veränderliche)
\item
  Integralrechnung
\item
  Lineare Algebra
\end{itemize}

\section*{Studien-/Prüfungsleistungen:}\label{studien-pruxfcfungsleistungen-11}

Klausur (60 min) sowie erfolgreiche Teilnahme am Praktikum als
Zulassungsvoraussetzung

\section*{Literatur:}\label{literatur-10}

\begin{itemize}
\tightlist
\item
  Skript unter \url{www.gm.fh-koeln.de/~konen}
\item
  Teschl, Gerald und Teschl, Susanne: ``Mathematik für Informatiker'',
  Springer Verlag, 4. Auflage, 2013
\item
  Hartmann, Peter: ``Mathematik für Informatiker-Ein praxisbezogenes
  Lehrbuch'' Vieweg Verlag, 475 Seiten, 3. Auflage 2006
\item
  Papula, Lothar: ``Mathematik für Ingenieure und Naturwissenschaftler''
  Vieweg Verlag, 14. Auflage, 2014
\item
  Stingl, Mathematik für Fachhochschulen, Hanser 2003
\end{itemize}

\chapter{Mathematik 2}\label{mathematik-2}

\begin{modulHead}
\textbf{Modulverantwortlich}: Prof.~Dr.~Wolfgang
Konen
\end{modulHead}
\begin{modulHead}
\textbf{Kürzel}:
MA2
\end{modulHead}
\begin{modulHead}
\textbf{Studiensemester}:
2
\end{modulHead}
\begin{modulHead}
\textbf{Sprache}:
deutsch
\end{modulHead}
\begin{modulHead}
\textbf{Kreditpunkte}:
8
\end{modulHead}
\begin{modulHead}
\textbf{Voraussetzungen nach
Prüfungsordnung}: Keine über die Zulassungsvorrausetzungen zum Studium
hinausgehenden. Der vorherige Besuch von Mathematik I ist sinnvoll, aber
keine zwingende
Voraussetzung.
\end{modulHead}
\begin{modulHead}
\textbf{Typ}:
Pflichtmodul
\end{modulHead}


\section*{Lehrform/SWS:}\label{lehrformsws-14}

6 SWS: Vorlesung 3 SWS; Praktikum 1 SWS; Übung 2 SWS

\section*{Arbeitsaufwand:}\label{arbeitsaufwand-13}

Gesamtaufwand 240 h, davon

\begin{itemize}
\tightlist
\item
  54 h Vorlesung
\item
  18 h Praktikum
\item
  36 h Übung
\item
  132 h Selbststudium
\end{itemize}

\section*{Angestrebte
Lernergebnisse:}\label{angestrebte-lernergebnisse-14}

Die Studierenden sollen die Fähigkeiten zur Analyse realer oder
geplanter Systeme entwickeln, indem sie praktische Aufgabenstellungen
aus dem Informatik-Umfeld in mathematische Strukturen abstrahieren und
lernen, selbstständig die Modellfindung und die Ergebnisbeurteilung
vorzunehmen. Dabei sollen die Anwendungsbezüge der Mathematik deutlich
werden, z.B. die Beziehungen diskreter Strukturen wie der Graphen zu
vielfältigen grundlegenden Datenstrukturen, die Statistik zur
Deskription und Beurteilung von Beobachtungen und die Analysis zur
Verarbeitung von Signalen und zur Lösung von mathematischen Modellen.

\section*{Inhalt:}\label{inhalt-14}

\begin{itemize}
\tightlist
\item
  Mehrdimensionale Differenzialrechnung,
\item
  Graphentheorie,
\item
  Kombinatorik, Wahrscheinlichkeitsrechnung und Statistik,
\item
  Komplexe Zahlen,
\item
  Differentialgleichungen.
\end{itemize}

\section*{Studien-/Prüfungsleistungen:}\label{studien-pruxfcfungsleistungen-12}

Klausur (60 min) sowie erfolgreiche Teilnahme am Praktikum als
Zulassungsvoraussetzung

\section*{Literatur:}\label{literatur-11}

\begin{itemize}
\tightlist
\item
  \begin{enumerate}
  \def\labelenumi{\alph{enumi}.}
  \setcounter{enumi}{18}
  \tightlist
  \item
    Literaturliste auf der Homepage \url{www.gm.fh-koeln.de/~konen}
  \end{enumerate}
\item
  Skript unter \url{www.gm.fh-koeln.de/~konen/Mathe2-SS}
\end{itemize}

\chapter{Mensch-Computer Interaktion}\label{mensch-computer-interaktion}

\begin{modulHead}
\textbf{Modulverantwortlich}: Prof.~Dr.~Gerhard
Hartmann
\end{modulHead}
\begin{modulHead}
\textbf{Kürzel}:
MCI
\end{modulHead}
\begin{modulHead}
\textbf{Studiensemester}:
2
\end{modulHead}
\begin{modulHead}
\textbf{Sprache}:
deutsch
\end{modulHead}
\begin{modulHead}
\textbf{Kreditpunkte}:
10
\end{modulHead}
\begin{modulHead}
\textbf{Voraussetzungen nach
Prüfungsordnung}: keine
\end{modulHead}
\begin{modulHead}
\textbf{Typ}:
Pflichtmodul
\end{modulHead}


\section*{Lehrform/SWS:}\label{lehrformsws-15}

Vorlesung und Übung

\section*{Arbeitsaufwand:}\label{arbeitsaufwand-14}

Gesamtaufwand 300 h, davon

\begin{itemize}
\tightlist
\item
  65h Vorlesung
\item
  65h Übung
\item
  170 h Selbststudium
\end{itemize}

\section*{Angestrebte
Lernergebnisse:}\label{angestrebte-lernergebnisse-15}

\begin{itemize}
\tightlist
\item
  Die Studierenden erwerben Grundkenntnisse in kognitions-, arbeits- und
  organisations-psychologischen Grundkonzepten und können diese auf
  Problemstellungen, im Kontext der Mensch-Computer Interaktion,
  anwenden.
\item
  Die Studierenden kennen Modelle, Methoden, Arbeits- und
  Dokumentationstechniken der Mensch-Computer Interaktion, können sie
  anwenden, kritisch diskutieren und für konkrete Aktivitäten in
  Entwicklungsprojekten unter Abwägung der Alternativen auswählen.
\item
  Sie kennen relevante internationale Normen und Standards, können sie
  anwenden und kritisch diskutieren .
\item
  Sie kennen methodische Ansätze benutzer- oder benutzungsorientierter
  Entwicklungsprozesse und können diese systematisch und iterativ auf
  die Konzeption, Realisation, Evaluation und das Redesign von
  interaktiven Systemen anwenden.
\item
  Zudem kennen sie Konzepte und Vorgehensmodelle für die Integration von
  Software- und Usability Engineering in einem Gesamtprozess und können
  diese in Entwicklungsprojekten anwenden.
\item
  Die Studierenden erlangen die Fähigkeit zum fachlichen Diskurs.
\end{itemize}

\section*{Inhalt:}\label{inhalt-15}

\begin{itemize}
\tightlist
\item
  kognitionspsychologische Grundlagen
\item
  Benutzermodellierung
\item
  Tätigkeitsmodellierung
\item
  Spezifikationsformen für Nutzungskontexte
\item
  Spezifikation von Nutzungsanforderungen
\item
  Interaktionsmodelle
\item
  Interaktionsmodalitäten und --kodalitäten
\item
  Vorgehensmodelle (human-centered, usability-engineering,
  usage-centered design)
\item
  Design-Prinzipien, -Pattern, -Guidelines, -Styleguides
\item
  Prototyping und Sketching
\item
  Evaluation
\end{itemize}

\section*{Studien-/Prüfungsleistungen:}\label{studien-pruxfcfungsleistungen-13}

schriftliche Modulprüfung

\section*{Medienformen:}\label{medienformen-8}

\begin{itemize}
\tightlist
\item
  Beamergestützte Vorlesung
\item
  Case Studies
\item
  Lehrfilme
\end{itemize}

\section*{Literatur:}\label{literatur-12}

\begin{itemize}
\tightlist
\item
  Dix, A.; Finlay, J.; Abowd, G. \& Beale, R.: Human-Computer
  Interaction. Harlow, Pearson, 2004 (3rd ed.),
\item
  Benyon, D., Turner, S. Turner, P. Designing Interactive Systems:
  People, Activities, Contexts, Technologies, Addison Wesley, 2005,
\item
  Anderson, J.R.: Kognitive Psychologie. Heidelberg, Springer, 2001 (3.
  Auflage).
\item
  Beyer H. \& Holtzblatt K.: Contextual Design: Defining
  Customer-Centered Systems. San Francisco Morgan Kaufmann, 1997.
\item
  Cockburn, A.: Writing Effective Use Cases. Boston, Addison-Wesley,
  2000.
\item
  Constantine, L.; Lockwood, L.: Software for Use, ACM Press, 1999.
\item
  Dumas, J.S. \& Redish, J.C.: A Practical Guide to Usability Testing.
  Exter, Intellect Books, 1999 (rev. edition).
\item
  Hacker, W.: Allgemeine Arbeitspsychologie. Bern, Huber, 1998.
\item
  Hackos, J. \& Redish, J.: User and Task Analysis for Interface Design.
  New York, Wiley, 1998.
\item
  Holtzblatt K.; Wendell, J.B. \& Wood, S.: Rapid Contextual Design. A
  How-to Guide to Key Techniques for User-Centered Design. San
  Francisco, Morgan Kaufmann, 2005.
\item
  Johnson, J.: GUI Bloopers. San Francisco, Morgan Kaufmann, 2000.
\item
  Kulak, D. \& Guiney, E.: Use Cases. Requirements in Context. Boston,
  Addison-Wesley, 2000.
\item
  Mayhew, D.: The Usability Engineering Lifecycle. A Practitioner´s
  Handbook for User Interface Design. San Francisco: Morgan Kaufmann,
  1999.
\item
  Nielsen, J. \& Mack, R.L. (eds.): Usability Inspection Methods.
  NewYork, Wiley, 1994.
\item
  Preece, J; Rogers, Y. \& Sharp, H.: Interaction Design. Beyond
  Human-Computer Interaction. NewYork, Wiley, 2002.
\item
  Rosson, M.B. \& Carroll, J.M.: Usability Engineering. Scenario-Based
  Development of Human-Computer Interaction. San Francisco, Morgan
  Kaufmann, 2002.
\item
  Snyder, C: Paper Prototyping. San Francisco, Morgan Kaufmann, 2003.
\item
  Ulich, E.: Arbeitspsychologie. Stuttgart, Schäffer-Poeschel, 2001
  (5.Auflage).
\end{itemize}

\chapter{Informatik, Recht und
Gesellschaft}\label{informatik-recht-und-gesellschaft}

\begin{modulHead}
\textbf{Modulverantwortlich}: Prof.~Dr.~Mario
Winter
\end{modulHead}
\begin{modulHead}
\textbf{Kürzel}:
IRG
\end{modulHead}
\begin{modulHead}
\textbf{Studiensemester}:
5
\end{modulHead}
\begin{modulHead}
\textbf{Sprache}:
deutsch
\end{modulHead}
\begin{modulHead}
\textbf{Kreditpunkte}:
5
\end{modulHead}
\begin{modulHead}
\textbf{Voraussetzungen nach
Prüfungsordnung}: keine
\end{modulHead}
\begin{modulHead}
\textbf{Typ}:
Pflichtmodul
\end{modulHead}


\section*{Lehrform/SWS:}\label{lehrformsws-16}

4 SWS: Vorlesung 2 SWS; Übung 2 SWS

\section*{Arbeitsaufwand:}\label{arbeitsaufwand-15}

Gesamtaufwand: 150 h, davon

\begin{itemize}
\tightlist
\item
  36 h Vorlesung
\item
  36 h Übung
\item
  78 h Selbststudium
\end{itemize}

\section*{Angestrebte
Lernergebnisse:}\label{angestrebte-lernergebnisse-16}

Informatikerinnen und Informatiker analysieren und konstruieren
sozio-technische Systeme und entwickeln dabei semiotische Artefakte wie
z.B. Spezifikationen, Programme und Handbücher. Die entwickelten Systeme
bilden einerseits soziale Wirklichkeit in vielfältiger Form ab und
ändern andererseits diese Wirklichkeit durch ihren Einsatz.

Die Studierenden sollen befähigt werden

\begin{itemize}
\tightlist
\item
  ethische und rechtliche Aspekte des Einsatzes von Informatik-Systemen
  zu charakterisieren und
\item
  ein kritisches Bewusstsein für die aktuellen Fragen des
  wechselseitigen Einflusses von Informatik und Gesellschaft zu
  entwickeln sowie
\item
  die Grundbegriffe des deutschen Privatrechts zu verstehen und sich im
  dazugehörigen Gesetzeswerk zu orientieren,
\item
  um die unterschiedlichen Wechselwirkungen zwischen Informatik-Systemen
  und ihrem Einsatzumfeld erkennen und bewerten und insbesondere im
  Bereich des Vertragsrechts selbständige Lösungsvorschläge erarbeiten
  zu können.
\end{itemize}

\section*{Inhalt:}\label{inhalt-16}

\subsection*{Informatik und
Gesellschaft}\label{informatik-und-gesellschaft}

Die Wechselwirkungen zwischen den von Informatikern entwickelten
Systemen und ihrem Einsatzumfeld werden in drei großen Themenblöcken
behandelt:

\begin{itemize}
\tightlist
\item
  Informatik und soziale Kontexte
\item
  Komplexität und Sicherheit in sozio-technischenen Systemen
\item
  Systemgestaltung und Verantwortung der Informatik.
\end{itemize}

Beispielhafte Inhalte:

\begin{itemize}
\tightlist
\item
  Geschichte der Informatik
\item
  Bildung und Wissenschaft
\item
  Wissenschaften und Gesellschaft
\item
  Digitale Medien und Internet
\item
  Datenschutz und Überwachungstechniken
\item
  Informatik und Gestaltung
\item
  partizipative Systemgestaltung
\item
  Open Source
\item
  Ethische Leitlinien für Informatiker
\item
  Normen und Standards
\item
  philosophische Aspekte der Informatik
\end{itemize}

\subsection*{Recht}\label{recht}

\begin{itemize}
\tightlist
\item
  Einführung in das deutsche Privatrecht, insbesondere in das BGB.
\item
  Schwerpunkt im Schuldrecht, hier insbesondere im Vertragsrecht.
\item
  Besondere Aspekte des Verbraucherschutzes und der inhaltlichen
  Gestaltung von Verträgen.
\item
  Im Allgemeinen Teil des BGB wird auf den Vertragsschluss, die
  Willenerklärung als rechtsgeschäftliches Gestaltungsmittel und die
  allgemeinen Anforderungen an die Vertragspartner eingegangen.
\end{itemize}

\section*{Studien-/Prüfungsleistungen:}\label{studien-pruxfcfungsleistungen-14}

\subsection*{Informatik und
Gesellschaft}\label{informatik-und-gesellschaft-1}

Präsentation im OpenSpace, Klausur (60 Min).

\subsection*{Recht}\label{recht-1}

Klausur (60 Min.)

\section*{Medienformen:}\label{medienformen-9}

Beamergestützte Vorträge

\section*{Literatur:}\label{literatur-13}

\subsection*{IUG}\label{iug}

\begin{itemize}
\tightlist
\item
  Sara Baase: A Gift of Fire. Social, Legal, and Ethical Issues in
  Computing. Prentice Hall, Upper Saddle River, 1997
\item
  A.F. Chalmers: Wege der Wissenschaft. 5. Aufl., Springer, Heidelberg,
  2001
\item
  D.M. Hester, P.J. Ford: Computers and Ethics in the Cyberage. Prentice
  Hall, Upper Saddle River, 2001
\item
  P. Gola, C. Klug: Grundzüge des Datenschutzrechts. C.H. Beck, 2003
\item
  M. Pierson, D. Seiler: Internet-Recht im Unternehmen.
  Beck-Rechtsberater im dtv, Deutscher Taschenbuch Verlag, München, 2002
\item
  \url{http://www.gi-ev.de} Arbeitskreis Informatik und Verantwortung,
  Ethische Leitlinien der GI
\item
  \url{http://www.bfd.bund.de} Der Bundesbeauftragte für den Datenschutz
\item
  \url{http://www.aktiv.org/DVD} Deutsche Vereinigung für Datenschutz
\item
  \url{http://www.big-brother-award.org} Überwachungsinformationen
\end{itemize}

\subsection*{Recht}\label{recht-2}

\begin{itemize}
\tightlist
\item
  Bürgerliches Gesetzbuch in der aktuellen Taschenbuchausgabe des dtv
\end{itemize}

\subsection*{Fakultativ}\label{fakultativ}

\begin{itemize}
\tightlist
\item
  Eugen Klunziger, Einführung in das Bürgerliche Recht, Verlag Vahlen
\item
  Norbert Ullrich, Wirtschaftsrecht für Betriebswirte, Verlag Neue
  Wirtschaftsbriefe
\end{itemize}

\chapter{Paradigmen der
Programmierung}\label{paradigmen-der-programmierung}

\begin{modulHead}
\textbf{Modulverantwortlich}: Prof.~Dr.~Christian
Kohls
\end{modulHead}
\begin{modulHead}
\textbf{Kürzel}:
PP
\end{modulHead}
\begin{modulHead}
\textbf{Studiensemester}:
3
\end{modulHead}
\begin{modulHead}
\textbf{Sprache}:
deutsch
\end{modulHead}
\begin{modulHead}
\textbf{Kreditpunkte}:
5
\end{modulHead}
\begin{modulHead}
\textbf{Voraussetzungen nach
Prüfungsordnung}: Keine über die Zulassungsvorrausetzungen
hinausgehenden
Vorraussetzungen
\end{modulHead}
\begin{modulHead}
\textbf{Typ}:
Pflichtmodul
\end{modulHead}


\section*{Lehrform/SWS:}\label{lehrformsws-17}

4 SWS: Vorlesung 2 SWS; Praktikum SWS; Übung 1SWS

\section*{Arbeitsaufwand:}\label{arbeitsaufwand-16}

Gesamtaufwand 150 h, davon

\begin{itemize}
\tightlist
\item
  36 h Vorlesung
\item
  18 h Praktikum
\item
  18 h Übung
\item
  78 h Selbststudium
\end{itemize}

\section*{Angestrebte
Lernergebnisse:}\label{angestrebte-lernergebnisse-17}

\begin{itemize}
\tightlist
\item
  Unterscheidung zwischen verschiedenen imperativen und deklarativen
  Programmierparadigmen kennen
\item
  Einordnung der Anwendbarkeit unterschiedlicher Programmierkonzepte
\end{itemize}

\section*{Inhalt:}\label{inhalt-17}

\begin{itemize}
\tightlist
\item
  Grundlagen von Programmiersprachen
\item
  Vergleich imperativer und deklarativer Paradigmen
\item
  prozedurale und objektorientierte Programmierung
\item
  funktionale Programmierung
\item
  Logikprogrammierung
\item
  Nebenläufigkeit
\item
  aspektorientierte Programmierung
\end{itemize}

\section*{Studien-/Prüfungsleistungen:}\label{studien-pruxfcfungsleistungen-15}

Klausur sowie erfolgreiche Teilnahme am Praktikum als
Prüfungsvorleistung

\section*{Medienformen:}\label{medienformen-10}

\begin{itemize}
\tightlist
\item
  Foliensammlung
\item
  Skript
\item
  Beispiellösungen
\end{itemize}

\section*{Literatur:}\label{literatur-14}

\begin{itemize}
\tightlist
\item
  Abelson, Sussman, Struktur und Interpretation von Computer
  Programmen,Springer-Verlag 2001
\item
  W.F. Clocksin, C.S. Mellish, Programming in Prolog, Springer-Verlag
  2003
\item
  Gamma, E., Helm, R., Johnson, R., \& Vlissides, J. (2015). Design
  patterns: Entwurfsmuster als Elemente wiederverwendbarer
  objektorientierter Software. Frechen: Mitp.
\item
  Odersky, Spoon, Venners, Programming in Scala, Artima Press 2011
\item
  Goetz, Bloch, Bowbeer, Lea, Java-Concurrency in Practise, Addison
  Wesley 2006
\end{itemize}

\chapter{Praxisprojekt}\label{praxisprojekt}

\begin{modulHead}
\textbf{Modulverantwortlich}: alle Informatik
Professoren
\end{modulHead}
\begin{modulHead}
\textbf{Kürzel}:
PP
\end{modulHead}
\begin{modulHead}
\textbf{Studiensemester}:
6
\end{modulHead}
\begin{modulHead}
\textbf{Sprache}:
deutsch
\end{modulHead}
\begin{modulHead}
\textbf{Kreditpunkte}:
10
\end{modulHead}
\begin{modulHead}
\textbf{Voraussetzungen nach
Prüfungsordnung}: alle Modulprüfungen außer Praxisprojekt,
Bachelorarbeit und Kolloquium
bestanden
\end{modulHead}
\begin{modulHead}
\textbf{Typ}:
Pflichtmodul
\end{modulHead}


\section*{Lehrform/SWS:}\label{lehrformsws-18}

Angeleitetes, eigenverantwortliches Arbeiten

\section*{Arbeitsaufwand:}\label{arbeitsaufwand-17}

300 h Projektarbeit

\section*{Angestrebte
Lernergebnisse:}\label{angestrebte-lernergebnisse-18}

Die Studierenden

\begin{itemize}
\tightlist
\item
  können Methoden und Techniken, die sie im Studium erlernt haben, in
  realitätsnahen Projekten weitgehend selbstständig anwenden
\item
  haben erste Erfahrungen mit der Selbststeuerung und proaktiven
  Kommunikation in einem Projekt mittlerer Größe und der Einordnung von
  Projektarbeit in betriebliche, gesellschaftliche und rechtliche
  Rahmenbedingungen gesammelt
\end{itemize}

\section*{Inhalt:}\label{inhalt-18}

Modulinhalte des ersten bis fünften Semesters anhand von realen
Anforderungen in einem praxisrelevanten Kontext anwenden und den
Studierenden durch die Betreuung des Dozenten an eine selbstständige
Projektdurchführung und Kommunikation heranführen. Das Praxisprojekt
kann entweder in einem Unternehmen oder in der Hochschule - dann
eingebettet in Forschungsprojekte - erfolgen.

\section*{Studien-/Prüfungsleistungen:}\label{studien-pruxfcfungsleistungen-16}

Schriftliche Ausarbeitung, Projektdokumentation

\chapter{Praxisprojektseminar}\label{praxisprojektseminar}

\begin{modulHead}
\textbf{Modulverantwortlich}: Prof.~Christian
Noss
\end{modulHead}
\begin{modulHead}
\textbf{Kürzel}:
PPS
\end{modulHead}
\begin{modulHead}
\textbf{Studiensemester}:
6
\end{modulHead}
\begin{modulHead}
\textbf{Sprache}:
deutsch
\end{modulHead}
\begin{modulHead}
\textbf{Kreditpunkte}:
5
\end{modulHead}
\begin{modulHead}
\textbf{Voraussetzungen nach
Prüfungsordnung}: alle Modulprüfungen außer Bachelorarbeit und
Kolloquium bestanden
\end{modulHead}
\begin{modulHead}
\textbf{Typ}:
Pflichtmodul
\end{modulHead}


\section*{Lehrform/SWS:}\label{lehrformsws-19}

Seminar

\section*{Arbeitsaufwand:}\label{arbeitsaufwand-18}

Gesamtaufwand 150 h, davon

\begin{itemize}
\tightlist
\item
  32 h Seminar
\item
  118 h Selbststudium
\end{itemize}

\section*{Angestrebte
Lernergebnisse:}\label{angestrebte-lernergebnisse-19}

Die Studierenden

\begin{itemize}
\tightlist
\item
  kennen Techniken wissenschaftlichen Arbeitens
\item
  haben erste Erfahrungen mit aktiver Fachkommunikation gesammelt
\item
  gewinnen einen ersten Überblick über das Spektrum von aktuellen Themen
  in der Medieninformatik
\item
  können eigene Projektergebnisse vor einem Fachpublikum in Vortrag und
  Diskussion darstellen und verteidigen
\end{itemize}

\section*{Inhalt:}\label{inhalt-19}

Das Praxisprojektseminar besteht aus

\begin{itemize}
\tightlist
\item
  Veranstaltungen in denen Techniken wissenschaftlichen Arbeitens
  vermittelt werden,
\item
  Fachvorträgen von Studierenden über Ihre Projektergebnisse.
\end{itemize}

\section*{Studien-/Prüfungsleistungen:}\label{studien-pruxfcfungsleistungen-17}

Seminarvortrag zur Praxisprojektarbeit

\section*{Literatur:}\label{literatur-15}

\begin{itemize}
\tightlist
\item
  Rechenberg: Technisches Schreiben (nicht nur) für Informatiker, 2.
  Aufl, Hanser Verlag 2003
\item
  M. Karmasin, R. Ribing: Die Gestaltung wissenschaftlicher Arbeiten, 2.
  Auflage WUV 2007
\end{itemize}

\chapter{Projektmanagement}\label{projektmanagement}

\begin{modulHead}
\textbf{Modulverantwortlich}: Prof.~Dr.~Holger
Günther, Prof.~Dr.~Mario
Winter
\end{modulHead}
\begin{modulHead}
\textbf{Kürzel}:
PM
\end{modulHead}
\begin{modulHead}
\textbf{Studiensemester}:
5
\end{modulHead}
\begin{modulHead}
\textbf{Sprache}:
deutsch
\end{modulHead}
\begin{modulHead}
\textbf{Kreditpunkte}:
5
\end{modulHead}
\begin{modulHead}
\textbf{Voraussetzungen nach
Prüfungsordnung}: Zulassungsbedingung - Abgeschlossenes Grundstudium;
Sonst keine besonderen
Voraussetzungen;
\end{modulHead}
\begin{modulHead}
\textbf{Typ}:
Pflichtmodul
\end{modulHead}


\section*{Kurzbeschreibung:}\label{kurzbeschreibung-2}

Managementaspekte der professionellen Entwicklung großer Softwaresysteme

\section*{Lehrform/SWS:}\label{lehrformsws-20}

4 SWS: Vorlesung 2 SWS, Übung 1 SWS, Praktikum 1 SWS; max. 6 Studierende
/ Praktikumsteam;

\section*{Arbeitsaufwand:}\label{arbeitsaufwand-19}

Gesamtaufwand 150 h, davon

\begin{itemize}
\tightlist
\item
  36 h Vorlesung
\item
  18 h Übung
\item
  18 h Praktikum
\item
  78 h Selbststudium
\end{itemize}

\section*{Angestrebte
Lernergebnisse:}\label{angestrebte-lernergebnisse-20}

Die Studierenden sollen befähigt werden,

\begin{itemize}
\tightlist
\item
  die grundlegenden Aufgaben des Projektmanagements, insb. in
  IT-Projekten, zu charakterisieren und durchzuführen
\item
  die Projektmanagement-Methoden, -Techniken und -Werkzeuge
  zielgerichtet einzusetzen
\item
  die erforderlichen soziologischen und kommunikativen Aspekte zu
  berücksichtigen,
\end{itemize}

um, mit dem Ziel einer menschengerechten und soziologisch fundierten
Menschenführung, eine wirkliche und optimale Produktivität bei komplexen
Projekten erreichen zu können.

\section*{Inhalt:}\label{inhalt-20}

Das Modul befasst sich mit den Managementaspekten der professionellen
Entwicklung großer Softwaresysteme.

Der Vorlesungsteil des Moduls gliedert sich in folgende Kapitel:

\begin{itemize}
\tightlist
\item
  Überblick -- Warum Projektmanagement?
\item
  Teamarbeit und Menschenführung (Kommunikation und Führung)
\item
  Kosten/Nutzen-Analysen und Entscheidungstechniken
\item
  Projektorganisation und Projektplanung (Aufbauorganisation,
  Ablauforganisation, Prozessmodellierung, iterative und agile
  Vorgehensmodelle, Netzplantechnik)
\item
  detaillierte Aufwandsschätzung und Projektcontrolling (Function Point
  Analysis, COCOMO, Risikomanagement, Projektpräsentationen)
\item
  Inhalte von PM-BOK (Project Management - Body of Knowledge)
\item
  Zusammenfassung und Prüfungsvorbereitung;
\end{itemize}

Damit die Studierenden die vorgestellten Methoden und Techniken zum
Management von Softwareprojekten anwenden, sowie besser analysieren und
bewerten können, werden in Projekt-Teams die in der Vorlesung
vermittelten Inhalte anhand eines Fallbeispiels eingesetzt. Dazu bilden
die Teilnehmenden Teams zu jeweils 6 Studierenden. Im Praktikum werden
folgende Bereiche vertieft:

\begin{itemize}
\tightlist
\item
  Kosten- Nutzenrechnung, Entscheidungstechniken
\item
  Aufbauorganisation
\item
  Aufwandsschätzung (Function-Point-Analyse, COCOMO);
\item
  Risikomanagement
\item
  Ablauf- und Ressourcenplanung (Netzplantechnik, Einsatz von
  PM-Software wie z.B. MS-Project)
\end{itemize}

\section*{Studien-/Prüfungsleistungen:}\label{studien-pruxfcfungsleistungen-18}

\begin{itemize}
\tightlist
\item
  Praktikum-Ausarbeitung
\item
  Vortrag
\item
  Mündliche Prüfung.
\end{itemize}

\section*{Medienformen:}\label{medienformen-11}

\begin{itemize}
\tightlist
\item
  Beamer-gestützte Vorlesungen (Folien in elektronischer Form im Netz);
\item
  Vertiefende Unterlagen sowie aktuelle Artikel (in elektronischer Form
  im Netz);
\item
  Praktika in Kleingruppen, um die erlernten Methoden und Techniken
  einzuüben und zu vertiefen (Seminarraum, Rechnerlabor);
\end{itemize}

\section*{Literatur:}\label{literatur-16}

\begin{itemize}
\tightlist
\item
  A. Buhl: Grundkurs Projektmanagement. Carl Hanser Verlag, München,
  2004
\item
  H.W. Wieczorrek, P. Mertens: Management von IT-Projekten Von der
  Planung zur Realisierung. 4. Aufl., Springer, Heidelberg, 2011
\item
  C. Aichele, M. Schönberger: IT-Projektmanagement. Springer Vieweg,
  2014
\item
  A. Henrich: Management von Softwareprojekten. Oldenbourg Verlag,
  München, 2002
\item
  H. Kerzner: Projektmanagement -- Ein systemorientierter Ansatz.
  mitp-Verlag, Bonn, 2003
\item
  T. DeMarco: Der Termin. Hanser, München, 1998
\end{itemize}

\chapter{Screendesign}\label{screendesign}

\begin{modulHead}
\textbf{Modulverantwortlich}: Prof.~Christian
Noss
\end{modulHead}
\begin{modulHead}
\textbf{Kürzel}:
SD
\end{modulHead}
\begin{modulHead}
\textbf{Studiensemester}:
3
\end{modulHead}
\begin{modulHead}
\textbf{Sprache}:
deutsch
\end{modulHead}
\begin{modulHead}
\textbf{Kreditpunkte}:
5
\end{modulHead}
\begin{modulHead}
\textbf{Voraussetzungen nach
Prüfungsordnung}: keine
\end{modulHead}
\begin{modulHead}
\textbf{Typ}:
Pflichtmodul
\end{modulHead}


\section*{Lehrform/SWS:}\label{lehrformsws-21}

4 SWS: Vorlesung 1 SWS; Seminar/Workshops 3 SWS; Projekt 1 SWS

\section*{Arbeitsaufwand:}\label{arbeitsaufwand-20}

Gesamtaufwand 150 h, davon

\begin{itemize}
\tightlist
\item
  40 h Seminar
\item
  80 h Projektarbeit
\item
  30 h Selbststudium
\end{itemize}

\section*{Angestrebte
Lernergebnisse:}\label{angestrebte-lernergebnisse-21}

Die Studierenden kennen wesentliche Begriffe der visuellen Kommunikation
und können diese anwenden um Briefings, Angebote oder Korrekturwünsche
im Design-Kontext zu verstehen oder zu verfassen.

Die Studierenden können Gestaltungslösungen und -kontexte analysieren,
argumentieren, diskutieren, dokumentieren und bewerten, um eigene
Lösungen innerhalb eines Gestaltungskontextes generieren zu können.

Die Studierenden können in einem gegebenen Gestaltungskontext, unter
Berücksichtigung von Gestaltungsregeln (Raster, Layout, Typographie,
etc.), eigene Gestaltungslösungen entwickeln, systematisch variieren und
argumentieren um gegebene funktionale und/oder kommunikative Ziele zu
adressieren.

\section*{Inhalt:}\label{inhalt-21}

\subsection*{Vorlesung:}\label{vorlesung}

\begin{itemize}
\tightlist
\item
  Design Basics
\item
  Axis Map \& Semantisches Differential
\item
  Kommunikationsmodelle
\item
  Visuelle Wahrnehmung
\item
  Benutzerziele
\item
  Corporate Identity
\item
  Orientierung, Hierarchisierung, Reduktion
\item
  Räumlichkeit
\item
  Gestaltgesetze
\item
  Farbe, Kontraste
\item
  Typographie, Textsatz
\item
  Proportion
\item
  Ordnung, visuelle Struktur, Flow \& Transistion
\item
  Gestaltungsziele, Gestaltungsprozess
\end{itemize}

\subsection*{Seminar:}\label{seminar}

\begin{itemize}
\tightlist
\item
  Designprojekte strukturieren
\item
  Layoutentwicklung mit Wireframes
\item
  Layoutentwicklung für verschiedene Endgeräte
\item
  Flow \& Transition
\item
  Typographie \& Textsatz
\item
  Designkonzepte analysieren \& bewerten
\item
  Variantenbildung
\item
  Modularisierung, Interface Inventar aufbauen \& visualisieren
\end{itemize}

\section*{Studien-/Prüfungsleistungen:}\label{studien-pruxfcfungsleistungen-19}

Projektpräsentationsprüfung

\section*{Medienformen:}\label{medienformen-12}

Beamergestützte Vorträge, Rechnergestützte Workshops

\section*{Literatur:}\label{literatur-17}

\begin{itemize}
\tightlist
\item
  Stapelkamp, Torsten: Informationsvisualisierung
\item
  Joachim Böhringer, Peter Bühler \& Patrick Schlaich: Kompendium der
  Mediengestaltung - Konzeption und Gestaltung für Digital- und
  Printmedien
\item
  Stapelkamp, Torsten: Screen- und Interfacedesign
\item
  Max Bollwage: Typografie kompakt
\item
  Kerstin Alexander: Kompendium der visuellen Information und
  Kommunikation
\item
  Maeda, John:Simplicity!: Die zehn Gesetze der Einfachheit
\item
  Lidwell, William; Holden, Kristina; Butler, Jill: Design: Die 100
  Prinzipien für erfolgreiche Gestaltung
\item
  Lewandowsky, Pina; Zeischegg, Francis: Visuelles Gestalten mit dem
  Computer
\item
  Koschembar, Frank: Grafik für Nicht-Grafiker
\end{itemize}

\chapter{Softwaretechnik}\label{softwaretechnik}

\begin{modulHead}
\textbf{Modulverantwortlich}: Prof.~Dr.~Mario
Winter
\end{modulHead}
\begin{modulHead}
\textbf{Kürzel}:
ST1
\end{modulHead}
\begin{modulHead}
\textbf{Studiensemester}:
4
\end{modulHead}
\begin{modulHead}
\textbf{Sprache}:
deutsch
\end{modulHead}
\begin{modulHead}
\textbf{Kreditpunkte}:
5
\end{modulHead}
\begin{modulHead}
\textbf{Voraussetzungen nach
Prüfungsordnung}: Keine über die Zulassungsbedingungen hinausgehenden
Voraussetzungen
\end{modulHead}
\begin{modulHead}
\textbf{Typ}:
Pflichtmodul
\end{modulHead}


\section*{Kurzbeschreibung:}\label{kurzbeschreibung-3}

Prinzipien, Methoden und Techniken der modellbasierten methodischen
objektorientierten Softwareentwicklung

\section*{Lehrform/SWS:}\label{lehrformsws-22}

4 SWS: Vorlesung 2 SWS; Parktikum 2 SWS

max. 15 Studierende/Praktikumsgruppe;

\section*{Arbeitsaufwand:}\label{arbeitsaufwand-21}

Gesamtaufwand 150 h, davon

\begin{itemize}
\tightlist
\item
  36 h Vorlesung
\item
  36 h Praktikum
\item
  78 h Selbststudium
\end{itemize}

\section*{Angestrebte
Lernergebnisse:}\label{angestrebte-lernergebnisse-22}

Die Studierenden sollen befähigt werden,

\begin{itemize}
\tightlist
\item
  zu abstrahieren, Modelle zu entwickeln, Unterschiede zwischen Modell
  und Realität zu beurteilen sowie
\item
  gegebene Modelle zu interpretieren, zu analysieren und zu bewerten,
\item
  um komplexe Systeme zu analysieren, im Team zu entwerfen und dabei im
  Rahmen methodischer Vorgehensweisen Techniken und Werkzeuge der
  objektorientierten Modellierung und Softwareentwicklung in den
  Aktivitäten Anforderungsermittlung, Softwarespezifizierung und Entwurf
  einzusetzen.
\end{itemize}

\section*{Inhalt:}\label{inhalt-22}

Die Vorlesung skizziert zunächst das Gesamtgebiet Softwaretechnik und
behandelt dann ausschließlich grundlegende „Informatikaspekte'' der
objektorientierten Softwareentwicklung. Als wesentliche Grundlage werden
die wichtigsten Elemente der Unified Modelling Language (UML)
vorgestellt und anhand kleinerer Beispiele erläutert. Danach werden
typische Aktivitäten der Softwareentwicklung besprochen, wobei die UML
als Modellierungssprache benutzt wird. Im Praktikum werden die Anwendung
der Modellierungselemente und die Durchführung der Aktivitäten in
Gruppenarbeit vertieft.

Das Modul gliedert sich in folgende Inhalte:

\begin{itemize}
\tightlist
\item
  (10\%) Softwareentwicklung im Überblick (Komplexität großer Software,
  Kernaktivitäten und unterstützende Aktivitäten);
\item
  (30\%) Die Modellierungssprache UML (Strukturmodellierung mit Objekt-
  und Klassendiagrammen, Funktionsmodellierung mit
  Anwendungsfalldiagrammen, Verhaltensmodellierung mit Sequenz-,
  Kommunikations- und Zustandsdiagrammen);
\item
  (50\%) Modellbasierte Softwareentwicklung (Anforderungsermittlung,
  Softwarespezifizierung und Architekturkonzeption, Entwurfskonzepte und
  Grobentwurf, Feinentwurf);
\item
  (10\%) Zusammenfassung und Ausblick (Modellgetriebene
  Softwareentwicklung);
\end{itemize}

\section*{Studien-/Prüfungsleistungen:}\label{studien-pruxfcfungsleistungen-20}

Klausur 120 Minuten

\section*{Medienformen:}\label{medienformen-13}

\begin{itemize}
\tightlist
\item
  Flipped-Classroom mit Diskussion und Übungen als Einzel- und
  Kleinstgrupen
\item
  e-Vorlesungen (Video-Clips und Folien in elektronischer Form zum
  Selbststudium);
\item
  Vertiefende Materialien in elektronischer Form (z.B. SWEBOK)
\item
  Praktika in Kleingruppen, um die erlernten Modelle und Methoden
  einzuüben und zu vertiefen (Seminarraum, Rechnerlabor); In den
  Praktika werden Modellierungs- und Entwicklungswerkzeuge eingesetzt.
\end{itemize}

\section*{Literatur:}\label{literatur-18}

\begin{itemize}
\tightlist
\item
  Helmut Balzert: Lehrbuch der Software-Technik Bd. I: Basiskonzepte und
  Requirements Engineering; Spektrum Akademischer Verlag, Heidelberg, 3.
  Aufl. 2009
\item
  Helmut Balzert: Lehrbuch der Software-Technik Bd. II: Entwurf,
  Implementierung, Installation und Betrieb; Spektrum Akademischer
  Verlag, Heidelberg, 3. Aufl. 2012
\item
  Helmut Balzert: Lehrbuch der Software-Technik Bd. III: Software
  Management; Spektrum Akademischer Verlag, Heidelberg, 2. Aufl. 2008
\item
  Martina Seidl et al.: UML@Classroom; dpunkt.Verlag, Heidelberg, 2012
\end{itemize}

Unterlagen/Videos: \url{http://www.uml.ac.at/lernen}

\begin{itemize}
\tightlist
\item
  Winter, M.: Methodische objektorientierte Softwareentwicklung.
  dpunkt.verlag, Heidelberg, 2005;
\item
  Chris Rupp et al.: UML 2 Glasklar. 4. Aufl., Carl Hanser Verlag,
  München, 2012
\item
  Jochen Ludewig, Horst Lichter: Software Engineering -- Grundlagen,
  Menschen, Prozesse, Techniken. 2. Aufl., dPunkt Verlag, Heidelberg,
  2011
\end{itemize}

\chapter{Theoretische Informatik 1}\label{theoretische-informatik-1}

\begin{modulHead}
\textbf{Modulverantwortlich}: Prof.~Dr.~Martin
Eisemann
\end{modulHead}
\begin{modulHead}
\textbf{Kürzel}:
TI1
\end{modulHead}
\begin{modulHead}
\textbf{Studiensemester}:
1
\end{modulHead}
\begin{modulHead}
\textbf{Sprache}:
deutsch
\end{modulHead}
\begin{modulHead}
\textbf{Kreditpunkte}:
5
\end{modulHead}
\begin{modulHead}
\textbf{Voraussetzungen nach
Prüfungsordnung}: Einfache Kenntnisse der naiven Mengenlehre, wie sie in
der Schule vermittelt und bei der mathematischen Begriffsbildung
verwendet werden.
\end{modulHead}
\begin{modulHead}
\textbf{Typ}:
Pflichtmodul
\end{modulHead}


\section*{Lehrform/SWS:}\label{lehrformsws-23}

4 SWS: Vorlesung 2 SWS; Übung 2 SWS

\section*{Arbeitsaufwand:}\label{arbeitsaufwand-22}

Gesamtaufwand 150 h, davon

\begin{itemize}
\tightlist
\item
  50h Vorlesung (inkl. Vor- und Nachbearbeitung)
\item
  50h Übung (inkl. Vor- und Nachbearbeitung)
\item
  50h Selbstlernphase
\end{itemize}

\section*{Angestrebte
Lernergebnisse:}\label{angestrebte-lernergebnisse-23}

\begin{itemize}
\tightlist
\item
  Grundsätzliches Ziel des Kurses ist eine Einführung in die Begriffe,
  Methoden, Modelle und Arbeitsweise der Theoretischen Informatik anhand
  der ausgewählten Teilgebiete.
\item
  Dabei lernen die Studierenden Probleme und Sachverhalte zu
  abstrahieren und zu modellieren (etwa logische und algebraische
  Kalküle, graphentheoretische Notationen, formale Sprachen und
  Automaten sowie spezielle Kalküle wie Petri-Netze)
\item
  Die Studierenden erwerben fundierte Kenntnisse der grundlegenden
  Themengebiete und eine wesentliche Basis und Vorbereitung für
  Veranstaltungen in höheren Semestern des Studiums.
\item
  In verschiedenen Grundlagengebieten der Informatik lernen die
  Studierenden Verfahrensweisen kennen, um den algorithmischen Kern
  eines Problems zu identifizieren und können passende Algorithmen
  entwerfen (Automaten, Turing Maschinen, Logik). Dabei können Sie
  bekannte Problemstellungen im Anwendungskontext erkennen und sind mit
  den zugehörigen Lösungsmustern vertraut (Modellierung mittels
  Automaten, Petri-Netzen, Boolescher Algebra, etc.).
\item
  Aufgaben zu den Lehrinhalten (s.u.) werden in kleinen Gruppen
  (Teamarbeit) selbständig gelöst. Die Lösungen sollen in den
  Übungsstunden vorgetragen und der Lösungsweg den Kommilitonen hierbei
  erläutert werden.
\end{itemize}

\section*{Inhalt:}\label{inhalt-23}

Dieses Grundlagenfach zur Informatik erstreckt sich über zwei Semester
(Theoretische Informatik 1 und Theoretische Informatik 2).

\begin{itemize}
\tightlist
\item
  Grundlagen: Mengen, Relationen, Graphen, Polynome; Zahlensysteme,
  Zahlendarstellung, Numerische Aspekte; Codierung, Informationstheorie
\item
  Logik und Boolesche Algebra: Aussagenlogik; Prädikatenlogik; Boolesche
  Algebra, Schaltnetze und Schaltwerke
\item
  Reguläre (Typ-3) Sprachen: Endliche Automaten, Reguläre Ausdrücke;
  Typ3-Grammatiken, Syntaxdiagramme; Chomsky-Hierarchie
\item
  Modellierung sequentieller und paralleler (Ausgabe-) Prozesse:
  Endliche Maschinen, Berechnungen; Automatennetze, Petri-Netze,
  Zelluläre Automaten
\item
  Kontextfreie (Typ-2) Sprachen: Kontextfreie Grammatiken, Chomsky- und
  Greibach-Normalform; Kellerautomaten; Anwendungen (Ableitungs- und
  Syntaxbäume, Syntax von Programmiersprachen, Backus-Naur-Form)
\item
  Kotextsensitive (Typ-1) und rekursiv aufzählende (Typ-0) Sprachen:
  Grammatiken, Monotonie, Normalform; Turingautomaten; Einführung in die
  Begriffe: Berechenbarkeit, Entscheidbarkeit und Komplexität
\end{itemize}

\section*{Studien-/Prüfungsleistungen:}\label{studien-pruxfcfungsleistungen-21}

Theoretische Informatik~1 + 2 als eine schriftliche Modulprüfung von 120
Min.

\section*{Medienformen:}\label{medienformen-14}

Vorlesung mit Übung

\section*{Literatur:}\label{literatur-19}

\begin{itemize}
\tightlist
\item
  Hoffmann, D. (2011): Theoretische Informatik, 2. Auflage
\item
  Brill, M. ( 2005 ): Mathematik für Informatiker.~Carl Hanser Verlag,
  München.
\item
  Vossen, G., Witt K. (2004): Grundlagen der Theoretischen Informatik
  mit Anwendungen.~3. Aufl.~ Vieweg \& Sohn, Braunschweig.
\item
  Hedtstück, U. ( 2004 ): Einführung in die Theoretische Informatik.
  Oldenbourg, München.
\item
  Kelly, J. ( 2003 ): Logik. Pearson Studium, München.
\item
  Kelch, R. ( 2003 ): Rechnergrundlagen. Von der Binärlogik zum
  Schaltwerk. Fachbuchverlag Leipzig im Carl Hanser Verlag.
\item
  Dean, N. ( 2003 ): Diskrete Mathematik. Pearson Studium. München.
\item
  Asteroth, A., Baier, C. (2002) Theoretische Informatik. Pearson
  Studium München
\item
  Hopcroft, J. E.~ et al. (2002): Einführung in die Automatentheorie,
  Formale Sprachen und Kom plexitätstheorie. Pearson Studium, München.
\item
  Meinel, C., Mundhenk, M. ( 2002 ): Mathematische Grundlagen der
  Informatik. B. G. Teubner, Stuttgart.
\item
  Ehrig, H. et al. (1999): Mathematisch-strukturelle Grundlagen der
  Informatik. Springer,~ Heidelberg.
\item
  Schöning, U. (1997): Theoretische Informatik - kurzgefaßt. 3. Aufl.
  Spektrum Akademischer Verlag, Heidelberg.
\item
  Schiffmann, W. und Schmitz, R. (1993): Technische Informatik 1.
  2.Aufl. Springer, Heidelberg.
\item
  Urbanski, K. und Woitowitz, R. (1993): Digitaltechnik.~~ BI-Wiss.-
  Verlag, Mannheim.
\item
  Beuth, K. (1992): Digitaltechnik. 9.Aufl.Vogel, Würzburg.
\item
  Morgenstern, B. (1992): Elektronik III, Digitale Schaltungen und
  Systeme. Vieweg \& Sohn, Braunschweig.
\item
  Rembold, U. et al.~ (1991): Einführung in die Informatik. 2.~ Aufl.
  Hanser, München.~~~~~~~~~~~~~~~~~~~~~~~~~~~~~~~
\item
  Rembold, U. et al.~ (1990): Aufgaben zur Informatik. Hanser, München.
\item
  Tietze, U. und Schenk, C. (1990): Halbleiter-Schaltungstechnik.
  9.Aufl. Springer, Berlin.
\end{itemize}

\chapter{Theoretische Informatik 2}\label{theoretische-informatik-2}

\begin{modulHead}
\textbf{Modulverantwortlich}: Prof.~Dr.~Martin
Eisemann
\end{modulHead}
\begin{modulHead}
\textbf{Kürzel}:
TI2
\end{modulHead}
\begin{modulHead}
\textbf{Studiensemester}:
2
\end{modulHead}
\begin{modulHead}
\textbf{Sprache}:
deutsch
\end{modulHead}
\begin{modulHead}
\textbf{Kreditpunkte}:
5
\end{modulHead}
\begin{modulHead}
\textbf{Voraussetzungen nach
Prüfungsordnung}: Einfache Kenntnisse der naiven Mengenlehre, wie sie in
der Schule vermittelt und bei der mathematischen Begriffsbildung
verwendet werden.
\end{modulHead}
\begin{modulHead}
\textbf{Typ}:
Pflichtmodul
\end{modulHead}


\section*{Lehrform/SWS:}\label{lehrformsws-24}

4 SWS: Vorlesung 2 SWS; Übung 2 SWS

\section*{Arbeitsaufwand:}\label{arbeitsaufwand-23}

Gesamtaufwand 150 h, davon

\begin{itemize}
\tightlist
\item
  50h Vorlesung (inkl. Vor- und Nachbearbeitung)
\item
  50h Übung (inkl. Vor- und Nachbearbeitung)
\item
  50h Selbstlernphase
\end{itemize}

\section*{Angestrebte
Lernergebnisse:}\label{angestrebte-lernergebnisse-24}

\begin{itemize}
\tightlist
\item
  Grundsätzliches Ziel des Kurses ist eine Einführung in die Begriffe,
  Methoden, Modelle und Arbeitsweise der Theoretischen Informatik anhand
  der ausgewählten Teilgebiete.
\item
  Dabei lernen die Studierenden Probleme und Sachverhalte zu
  abstrahieren und zu modellieren (etwa logische und algebraische
  Kalküle, graphentheoretische Notationen, formale Sprachen und
  Automaten sowie spezielle Kalküle wie Petri-Netze)
\item
  Die Studierenden erwerben fundierte Kenntnisse der grundlegenden
  Themengebiete und eine wesentliche Basis und Vorbereitung für
  Veranstaltungen in höheren Semestern des Studiums.
\item
  In verschiedenen Grundlagengebieten der Informatik lernen die
  Studierenden Verfahrensweisen kennen, um den algorithmischen Kern
  eines Problems zu identifizieren und können passende Algorithmen
  entwerfen (Automaten, Turing Maschinen, Logik). Dabei können Sie
  bekannte Problemstellungen im Anwendungskontext erkennen und sind mit
  den zugehörigen Lösungsmustern vertraut (Modellierung mittels
  Automaten, Petri-Netzen, Boolescher Algebra, etc.).
\item
  Aufgaben zu den Lehrinhalten (s.u.) werden in kleinen Gruppen
  (Teamarbeit) selbständig gelöst. Die Lösungen sollen in den
  Übungsstunden vorgetragen und der Lösungsweg den Kommilitonen hierbei
  erläutert werden.
\end{itemize}

\section*{Inhalt:}\label{inhalt-24}

Dieses Grundlagenfach zur Informatik erstreckt sich über zwei Semester
(Theoretische Informatik 1 und Theoretische Informatik 2).

\begin{itemize}
\tightlist
\item
  Grundlagen: Mengen, Relationen, Graphen, Polynome; Zahlensysteme,
  Zahlendarstellung, Numerische Aspekte; Codierung, Informationstheorie
\item
  Logik und Boolesche Algebra: Aussagenlogik; Prädikatenlogik; Boolesche
  Algebra, Schaltnetze und Schaltwerke
\item
  Reguläre (Typ-3) Sprachen: Endliche Automaten, Reguläre Ausdrücke;
  Typ3-Grammatiken, Syntaxdiagramme; Chomsky-Hierarchie
\item
  Modellierung sequentieller und paralleler (Ausgabe-) Prozesse:
  Endliche Maschinen, Berechnungen; Automatennetze, Petri-Netze
\item
  Kontextfreie (Typ-2) Sprachen: Kontextfreie Grammatiken, Chomsky- und
  Greibach-Normalform; Kellerautomaten; Anwendungen (Ableitungs- und
  Syntaxbäume, Syntax von Programmiersprachen, Backus-Naur-Form)
\item
  Kotextsensitive (Typ-1) und rekursiv aufzählende (Typ-0) Sprachen:
  Grammatiken, Monotonie, Normalform; Turingautomaten; Einführung in die
  Begriffe: Berechenbarkeit, Entscheidbarkeit und Komplexität
\end{itemize}

\section*{Studien-/Prüfungsleistungen:}\label{studien-pruxfcfungsleistungen-22}

Theoretische Informatik~1 + 2 als eine Modulprüfung von 120 Min.

\section*{Medienformen:}\label{medienformen-15}

Vorlesung und praktische Übungen

\section*{Literatur:}\label{literatur-20}

\begin{itemize}
\tightlist
\item
  Hoffmann, D. (2011): Theoretische Informatik, 2. Auflage
\item
  Brill, M. ( 2005 ): Mathematik für Informatiker.~Carl Hanser Verlag,
  München.
\item
  Vossen, G., Witt K. (2004): Grundlagen der Theoretischen Informatik
  mit Anwendungen.~3. Aufl.~ Vieweg \& Sohn, Braunschweig.
\item
  Hedtstück, U. ( 2004 ): Einführung in die Theoretische Informatik.
  Oldenbourg, München.
\item
  Kelly, J. ( 2003 ): Logik. Pearson Studium, München.
\item
  Kelch, R. ( 2003 ): Rechnergrundlagen. Von der Binärlogik zum
  Schaltwerk. Fachbuchverlag Leipzig im Carl Hanser Verlag.
\item
  Dean, N. ( 2003 ): Diskrete Mathematik. Pearson Studium. München.
\item
  Asteroth, A., Baier, C. (2002) Theoretische Informatik. Pearson
  Studium München
\item
  Hopcroft, J. E.~ et al. (2002): Einführung in die Automatentheorie,
  Formale Sprachen und Kom plexitätstheorie. Pearson Studium, München.
\item
  Meinel, C., Mundhenk, M. ( 2002 ): Mathematische Grundlagen der
  Informatik. B. G. Teubner, Stuttgart.
\item
  Ehrig, H. et al. (1999): Mathematisch-strukturelle Grundlagen der
  Informatik. Springer,~ Heidelberg.
\item
  Schöning, U. (1997): Theoretische Informatik - kurzgefaßt. 3. Aufl.
  Spektrum Akademischer Verlag, Heidelberg.
\item
  Schiffmann, W. und Schmitz, R. (1993): Technische Informatik 1.
  2.Aufl. Springer, Heidelberg.
\item
  Urbanski, K. und Woitowitz, R. (1993): Digitaltechnik.~~ BI-Wiss.-
  Verlag, Mannheim.
\item
  Beuth, K. (1992): Digitaltechnik. 9.Aufl.Vogel, Würzburg.
\item
  Morgenstern, B. (1992): Elektronik III, Digitale Schaltungen und
  Systeme. Vieweg \& Sohn, Braunschweig.
\item
  Rembold, U. et al.~ (1991): Einführung in die Informatik. 2.~ Aufl.
  Hanser, München.~~~~~~~~~~~~~~~~~~~~~~~~~~~~~~~
\item
  Rembold, U. et al.~ (1990): Aufgaben zur Informatik. Hanser, München.
\item
  Tietze, U. und Schenk, C. (1990): Halbleiter-Schaltungstechnik.
  9.Aufl. Springer, Berlin.
\end{itemize}

\chapter{Visual Computing}\label{visual-computing}

\begin{modulHead}
\textbf{Modulverantwortlich}: Prof.~Hans Kornacher,
Prof.~Dr.~Martin
Eisemann
\end{modulHead}
\begin{modulHead}
\textbf{Kürzel}:
VC
\end{modulHead}
\begin{modulHead}
\textbf{Studiensemester}:
4
\end{modulHead}
\begin{modulHead}
\textbf{Sprache}:
deutsch
\end{modulHead}
\begin{modulHead}
\textbf{Kreditpunkte}:
20
\end{modulHead}
\begin{modulHead}
\textbf{Voraussetzungen nach
Prüfungsordnung}: keine
\end{modulHead}
\begin{modulHead}
\textbf{Typ}:
Vertiefungsmodul
\end{modulHead}


\section*{Kurzbeschreibung}\label{kurzbeschreibung-4}

Das Modul „Visual Computing'' im Medieninformatik Bachelor beschäftigt
sich mit der Erzeugung und Verarbeitung visueller Informationen, sowohl
in realen als auch computergenerierten Szenarien.

Ziel dieses Moduls ist es den Studierenden eine fachlich fundierte,
praktische, sowie theoretische Grundlage im Umgang mit audiovisuellen
Medien zu geben. Dabei wird sowohl auf die technische Seite (technischen
Grundlagen der Video- und Fernsehtechnik) eingegangen als auch auf die
algorithmische (computergenerierte Bildsynthese, Gameentwicklung).

Das Modul ist aus vier Teilbereichen aufgebaut, von denen zwei
verpflichtend sind und zwei weitere aus einem Wahlkatalog gewählt werden
können.

Die beiden Pflichtkurse schaffen ein Fundament, was es erlaubt innerhalb
der beiden verbliebenen Kurse, im Gesamtumfang von 10 CP, tiefer in die
jeweilige Spezialisierung einzutauchen. Dabei gibt es grundsätzlich die
Möglichkeit sich in Richtung Fernseh- und Videoproduktion oder
Gameentwicklung zu vertiefen.

Die Kurse werden nach Verfügbarkeit angeboten.

Die Kurse sind in der Regel projektbasiert aufgebaut, so dass sowohl
theoretischer Hintergrund als auch praxisnahes Wissen vermittelt wird
und zur Anwendung kommt.

\section*{Lehrform/SWS:}\label{lehrformsws-25}

Vorlesung, Praktikum / Projekt, Übung mit kursabhängigen Schwerpunkten

\section*{Lehrveranstaltungen}\label{lehrveranstaltungen}

Pflichtbereich im Gesamtumfang von 10 CP:

\begin{itemize}
\tightlist
\item
  Audiovisuelle Medientechnik
\item
  Computergrafik und Animation
\end{itemize}

Aus den folgenden Wahlkatalogen sind zwei weitere Kurse zu jeweils 5CP
zu belegen, welche nach Verfügbarkeit angeboten werden.

Wahlkatalog Film/Video:

\begin{itemize}
\tightlist
\item
  Audiovisuelles Medienprojekt 2
\item
  Visuelle Effekte und Animation
\end{itemize}

Wahlkatalog Game Development:

\begin{itemize}
\tightlist
\item
  Crossplatform Game Development mit Unity3D
\item
  Prozedurale Generierung virtueller Welten
\end{itemize}

\section*{Arbeitsaufwand:}\label{arbeitsaufwand-24}

\begin{itemize}
\tightlist
\item
  Audiovisuelle Medientechnik: Gesamtaufwand 150 h, davon - 50h
  Vorlesung (inkl. Vor- und Nachbearbeitung) - 50h Praktikum / Projekt -
  50h Selbstlernphase
\item
  Computergrafik und Animation: Gesamtaufwand 150 h, davon - 50h
  Vorlesung (inkl. Vor- und Nachbearbeitung) - 50h Praktikum / Projekt -
  50h Selbstlernphase
\item
  Audiovisuelles Medienprojekt 2: Gesamtaufwand 150 h, davon - 50h
  Vorlesung (inkl. Vor- und Nachbearbeitung) - 50h Praktikum / Projekt -
  50h Selbstlernphase
\item
  Visuelle Effekte und Animation: Gesamtaufwand 150 h, davon - 50h
  Vorlesung (inkl. Vor- und Nachbearbeitung) - 50h Praktikum / Projekt -
  50h Selbstlernphase
\item
  Crossplatform Game Development mit Unity3D: Gesamtaufwand 150 h, davon
  - 50h Vorlesung (inkl. Vor- und Nachbearbeitung) - 60h Praktikum /
  Projekt - 40h Selbstlernphase
\item
  Prozedurale Generierung virtueller Welten: Gesamtaufwand 150 h, davon
  - 40 h Vorlesung (inkl. Vor- und Nachbereitung), - 40 h Live-Coding
  und Übungen - 40 h Selbstlernphase - 30 h Begleitetes Abschlussprojekt
\end{itemize}

\section*{Angestrebte
Lernergebnisse:}\label{angestrebte-lernergebnisse-25}

\subsection*{Audiovisuelle
Medientechnik}\label{audiovisuelle-medientechnik}

Die Studierenden sollen durch dieses Modul dazu befähigt werden, auf
Basis der technischen Grundlagen der Video- und Fernsehtechnik
weitergehende Fragestellungen selbstständig zu erarbeiten und sich so
auch zukünftige technische Entwicklungen autonom erschließen zu können.

Neben der Entwicklung und Förderung dieser Fachkompetenz ist die
Initiierung der Methodenkompetenz eine wichtige Säule des
Vorlesungsmoduls. Unter Methodenkompetenz ist hier die
Selbstorganisation im Sinne von wissenschaftlicher Fragestellung an
einen Themenkomplex und ein strukturiertes Vorgehen in der Erarbeitung
eines Lösungsansatzes zu verstehen. Ziel ist es, das Wissen aus
verschiedenen Bereichen, wie Kerninformatik, Internet- und
Webtechnologien und benachbarten Wissenschaften mit der in diesem Modul
unterrichteten Medientechnologien zu kombinieren und in die
Medienproduktion zu integrieren.

Gerade der Umgang mit Technologien und Methoden aus der Film- und
Fernsehproduktion erweitert den Erfahrungshorizont der Studierenden über
den bekannten Themenbereich der Kerninformatik hinaus und legt ihnen
eine Einarbeitung in informatikfremde Sachverhalte und technologische
Problemstellungen und deren Lösungsmethoden nahe.

Pragmatisches Ziel ist es, in den unterschiedlichsten Berufsfeldern
audiovisueller Medien die Entwicklung und den Einsatz digitaler
Medientechnik zu beraten, zu planen, durchzuführen oder zu verantworten.

\subsection*{Computergrafik und
Animation}\label{computergrafik-und-animation}

Die Grundlagen der zwei- und insbesondere der dreidimensionalen
Computergraphik und Animation stellen ein hervorragendes Paradigma zur
Vermittlung zentraler Inhalte und Kompetenzen der Medieninformatik dar.

Den Studierenden wird deutlich, wie der Bogen von den abstrakten,
geometrischen und algorithmischen Fakten zu den pragmatischen
Gegebenheiten der Computergraphik-Hardware gespannt ist.

Sie erkennen die Zusammenhänge zwischen Grundlagenvorlesungen
(Mathematik, Algorithmen, Programmierung) und der Gestaltung von
Schnittstellen und Oberflächen und werden so für die jeweiligen Inhalte
zusätzlich motiviert.

Dabei lernen Sie, im Kontext der Computergrafik, Verfahrensweisen
kennen, um den algorithmischen Kern eines Problems zu identifizieren und
können Algorithmen entwerfen, verifizieren und bzgl. ihres
Ressourcenbedarfs bewerten.

Sie erwerben die Fähigkeit, aktuelle technologische Entwicklungen im
Medieninformatik-Kontext zu bewerten und Trends einzuordnen.

Nach Abschluss des Moduls besitzen die Studierenden grundlegende
Kenntnisse über Architektur und Programmierung moderner Graphikhardware,
sowie deren Anwendung in konkreten Problemstellungen und
Anwendungskontexten.

Am Beispiel von OpenGL und der Rendering-Pipeline lernen die
Studierenden Problemstellungen im Anwendungskontext zu erkennen und sind
mit den zugehörigen Lösungsmustern durch praktische Programmierung
vertraut.

Das erlernte Wissen und die erlernten Kenntnisse in der Soft- und
Grafikhardware-Architektur ermöglicht es erfolgreichen Teilnehmern,
anschließend Echtzeit-Visualisierungen mit OpenGL zu implementieren und
somit mit einer modernen, plattformunabhängigen API umzugehen, die
flexibel an bestehende Anforderungen angepasst werden kann. Zudem haben
Sie die Fähigkeit hochparallele Algorithmen auf der Graphikkarte zu
entwerfen und auszuführen.

Dabei beherrschen die Studierenden nach Abschluss des Moduls die
Fähigkeit abstrakte Szenen- und Objektbeschreibungen zu erstellen und
darzustellen, sowie sich in vorhandenen Quelltext einzuarbeiten und
diesen sinnvoll weiter zu entwickeln.

Die Inhalte des Moduls befähigen die Studierenden die grundlegenden
Algorithmen und Datenstrukturen der Echtzeit-Computergraphik zu
beherrschen.

Die Studierenden können ihr erworbenes Können und Wissen zur
Implementierung einer eigenen Game/Visualisierungs-Engine einsetzen.
Dies zeigen Sie durch Umsetzung eines eigenen Projektes in Kleingruppen,
wo sie zusätzlich lernen mündlich überzeugend zu präsentieren,
abweichende Positionen zu erkennen und in eine sach- und
interessengerechte Lösung zu integrieren. Sie zeigen dadurch, dass Sie
in der Lage sind sich selbstständig neues Wissen anzueigenen und zu
erkennen, welches Wissen relevant ist, können mediengestalterische
Grundkompetenzen anwenden und besitzen aktive Vokabularien zur
Beschreibung und Realisierung angemessener Konzeptionen. Zudem können
sie die Realisationen bezüglich der Zielsetzungen kritisch diskutieren.

\subsection*{Audiovisuelles Medienprojekt
2}\label{audiovisuelles-medienprojekt-2}

Die praktische Umsetzung des Vorlesungsstoffes, die Kommunikation und
Zusammenarbeit im Team über Themenbereiche dieses Faches und die
Präsentation von eigenen Projekten und Untersuchungsergebnissen sind die
Lernziele des Moduls Audiovisuelles Medienprojekt 2. Neben dieser
formulierten Fachkompetenz, Methodenkompetenz und
Kommunikationskompetenz stehen gerade die sogenannten Softskills
Teamfähigkeit und Kommunikationsfähigkeit im Focus der Ausbildung in
diesem Modul.

Die Studierenden kennen über die grundlegenden Erzählformen
audiovisueller Medien hinaus spezielle Formate wie Spielfilm, Imagefilm
und Studioproduktion und haben dabei folgende Fertigkeiten: Sie können
eigene audiovisuelle Erzählformen auf der Basis dieser Erzählmuster
entwickeln und sind befähigt zur Analyse, zur Diskussion und zur
kritischen Betrachtung audiovisueller Medieninhalte.

Pragmatisches Ziel ist es, in den unterschiedlichsten Berufsfeldern
digitaler audiovisueller Medien die Entwicklung und den Einsatz
audiovisuellen Content zu beraten, zu planen, durchzuführen oder zu
verantworten.

\subsection*{Visuelle Effekte und
Animation}\label{visuelle-effekte-und-animation}

Die Studierenden kennen die grundlegenden Produktionsschritte und
Abläufe einer Film- und TV-Produktion mit visuellen Effekten sowie die
in diesem Zusammenhang eingesetzten Softwaretools.

Sie haben die Fertigkeit, spezifische Fragestellungen der Umsetzung
visueller, computerbasierter Effekte und der damit zusammenhängenden
Bildbearbeitung zu bearbeiten und fallbezogene individuelle Lösungen zu
entwickeln.

Unter Entwicklungs- und Methodenkompetenz auf dem Gebiet der Visual
Effects ist die Fähigkeit zu verstehen, eigene und für den jeweiligen
Anwendungsfall auch eventuell neue Lösungsansätze zu entwickeln, bei
denen die unterschiedlichen Methoden der Visual Effects-Ausführung und
-Bearbeitung zum Einsatz kommen. Nachdem die Planung, Durchführung und
die Bearbeitung von Projekten auf dem Gebiet der Film- und TV-Produktion
mit visuellen Effekten in der Regel im kleinen Team erfolgt sind gerade
die Softskills der Teamkompetenz und der Organisationskompetenz von
großer Wichtigkeit in diesem Modul.

Berufsbilder, die von diesem Modul angesprochen werden, sind zum einen
in der Visual-Effects-spezifischen Softwareentwicklung, als auch im
Anwendungskontext zu finden: So zum Beispiel in der Planung,
Organisation, Durchführung und Verantwortung von VFX-Projekten.

\subsection*{Cross-Platform Game Development mit Unity
3D}\label{cross-platform-game-development-mit-unity-3d}

Die Studierenden kennen wesentliche Konzepte und Technologien des Game
Developments mit Unity 3D und können diese anwenden, um eigenständig im
Team Interaktive Applikationen zu konzipieren, realisieren und
optimieren.

Die Studierenden kennen die grundlegenden Möglichkeiten von Game Engine
Frameworks und sind in der Lage diese kritisch zu beurteilen und auf
Basis der Anforderungen eines konkreten Projekts die
Umsetzungsmöglichkeiten und Vorgehen zu evaluieren und entsprechende
Strategien zu entwickeln, sowie kritisch die benötigten Bibliotheken und
Komponenten auszuwählen und diese Wahl zu begründen.

Die Kursteilnehmer sammeln im Rahmen ihres eigenständig entwickelten
Projektes Erfahrungen in der Entwicklung von Kleinprojekten bis
mindestens zum Grad einer spielbaren Alphaversion oder eines Prototypen.

\subsection*{Prozedurale Generierung virtueller
Welten}\label{prozedurale-generierung-virtueller-welten}

Die Studierenden haben die Möglichkeit ihr Wissen über
3D-Computergrafik, 3D-Geometrie und Programmierung zu vertiefen und
praktisch anzuwenden. Durch den Einsatz von Unity als
Crossplatform-Game-Development-Tool können die Studierenden ihre
Erfahrung mit einer aktuellen Game-Engine und der dazugehörigen
Entwicklungsumgebung vertiefen und werden befähigt diese um
Funktionalitäten zu erweitern.

Die Inhalte des Moduls befähigen die Studierenden die grundlegenden
Algorithmen und Datenstrukturen der Echtzeit-Computergrafik mit der
Game-Engine Unity zu beherrschen.

Das Modul ist daher geeignet das Wissen aus den Modulen „Computergrafik
und Animation``, sowie''Cross-Platform Game Development mit Unity 3D''
zu vertiefen und zu erweitern.

Konzepte aus der 3D-Computergrafik und der prozeduralen Generierung
werden in diesem sehr praktisch ausgelegten Kurs implementiert und die
relevanten Grundlagen vermittelt, die es den Studierenden ermöglichen
eigenständig Anforderungen für Projekte mit prozeduralen Techniken zu
entwickeln und diese praktisch umzusetzen und kritisch zu evaluieren.

Nach Abschluss des Moduls besitzen die Studierenden grundlegende
Kenntnisse über verschiedene Techniken der prozeduralen Generierung zur
Erstellung von 3D Content für Visualisierungen, Simulationen und Spiele,
sowie die Fähigkeiten diese einzusetzen, sowie selbst zu entwickeln.

Je nach gewählten Vertiefungskursen entwickeln die Studierenden
Fähigkeiten zur selbstverantwortlichen Durchführung von Projekten im
Bereich Gamedevelopment, dreidimensionaler Darstellung virtueller
Szenen, Film- und Fernsehtechnik, sowie Visueller Effekte.

Im Detail:

\section*{Inhalt:}\label{inhalt-25}

\subsection*{Audiovisuelle
Medientechnik}\label{audiovisuelle-medientechnik-1}

\begin{itemize}
\tightlist
\item
  Grundlagen der Fernsehtechnik
\item
  Digitale Fernsehtechnik
\item
  HD-Technik
\item
  Videodatenreduktion
\item
  Bildwandler
\item
  Das optische System der Videokamera
\item
  Signalverarbeitung in der Videokamera
\item
  Signalaufzeichnung
\item
  Elektroakustik
\item
  Bildwiedergabesysteme
\item
  Lichttechnik und Beleuchtung
\end{itemize}

\subsection*{Computergrafik und
Animation}\label{computergrafik-und-animation-1}

\begin{itemize}
\tightlist
\item
  Graphikhardware,
\item
  OpenGL
\item
  Transformationen und homogene Koordinaten
\item
  Interpolation
\item
  Kameramodelle
\item
  Clipping
\item
  Shaderprogrammierung
\item
  Animation
\item
  Texturierung
\item
  Fortgeschrittene Algorithmen (Schatten, Reflexionen, Bump-, Normal-,
  Parallax-, Relief-Mapping, Globale Beleuchtung, Deferred Shading)
\item
  Perzeption
\item
  Grundlagen des Ray Tracings
\end{itemize}

\subsection*{Audiovisuelles Medienprojekt
2}\label{audiovisuelles-medienprojekt-2-1}

\begin{itemize}
\tightlist
\item
  Vertiefung der Video- und Audioaufnahmetechnik
\item
  Verschiedene Dramaturgiemodelle
\item
  Drehbuch, Auflösung, Storyboard
\item
  Schnitt und Montage
\item
  Liveproduktion im Studio
\item
  Medienproduktion in den Formaten Spielfilm, Imagefilm und
  Studioproduktion
\end{itemize}

\subsection*{Visuelle Effekte und
Animation}\label{visuelle-effekte-und-animation-1}

\begin{itemize}
\tightlist
\item
  Storyboard
\item
  Kalkulation
\item
  Produktionabläufe
\item
  Keyverfahren mit Green- und Bluescreen
\item
  Compositing
\item
  Umgang mit Bild-/Videobearbeitungswerkzeugen
\end{itemize}

\subsection*{Cross-Platform Game Development mit Unity
3D}\label{cross-platform-game-development-mit-unity-3d-1}

\begin{itemize}
\tightlist
\item
  Aufbau einer Game Engine
\item
  Gameobjects
\item
  Game Physics
\item
  Interaktion
\item
  Spielmechaniken
\end{itemize}

\subsection*{Prozedurale Generierung virtueller
Welten}\label{prozedurale-generierung-virtueller-welten-1}

\begin{itemize}
\tightlist
\item
  Einführung in die Game-Engine Unity
\item
  Primitive und Mesh-Datenstrukturen
\item
  UV-Mapping und Texturierungstechniken/Materialien
\item
  Prozedurale Texturgenerierung
\item
  Parametrisierung von 3D-Modellen
\item
  Kurven und Flächen
\item
  Height-Maps
\item
  L-Systeme und „Turtle``-Grafik-Renderer
\item
  Triangulations-Algorithmen für Polygone
\item
  Voxel-Terrain-Generierung
\item
  Grundlagen 3D-Geometrie
\item
  Erweiterung des Unity-Editors
\end{itemize}

\section*{Studien-/Prüfungsleistungen:}\label{studien-pruxfcfungsleistungen-23}

\subsection*{Audiovisuelle
Medientechnik}\label{audiovisuelle-medientechnik-2}

Schriftliche Prüfung von 90 Minuten

\subsection*{Computergrafik und
Animation}\label{computergrafik-und-animation-2}

Die erfolgreiche Teilnahme an den Praktikas ist Voraussetzung für die
Klausur (120 Minuten) oder mündliche Prüfung

\subsection*{Audiovisuelles Medienprojekt
2}\label{audiovisuelles-medienprojekt-2-2}

Projektarbeit und schriftliche Ausarbeitung

\subsection*{Visuelle Effekte und
Animation}\label{visuelle-effekte-und-animation-2}

Projektarbeit und schriftliche Ausarbeitung

\subsection*{Cross-Platform Game Development mit Unity
3D}\label{cross-platform-game-development-mit-unity-3d-2}

Präsentation und Dokumentation eines eigenständig entwickelten Projekts

\subsection*{Prozedurale Generierung virtueller
Welten}\label{prozedurale-generierung-virtueller-welten-2}

Die erfolgreiche Teilnahme am Abschlussprojekt (eigenständiges Projekt,
auch in Kleingruppen möglich) und Fachgespräch

Teilprüfungen in den jeweiligen Kursen.

\section*{Medienformen:}\label{medienformen-16}

Beamergestützte Vorträge, Rechnergestützte Workshops

\section*{Literatur:}\label{literatur-21}

\subsection*{Audiovisuelle
Medientechnik}\label{audiovisuelle-medientechnik-3}

\begin{itemize}
\tightlist
\item
  Schmidt Ulrich, Professionelle Videotechnik, Springer-Verlag Berlin
  Heidelberg New York 2013, ISBN 978-3-642-38992-4
\item
  Johannes Webers, Film- und Fernsehtechnik, Franzis Verlag, Poing 2000,
  ISBN 3-7723-7116-7
\item
  Möllering, Slansky, Handbuch der professionellen Videoaufnahme Edition
  Filmwerkstatt, Essen, 1993, ISBN 3 - 9 802 581 - 3 - 0
\end{itemize}

\subsection*{Computergrafik und
Animation}\label{computergrafik-und-animation-3}

\begin{itemize}
\tightlist
\item
  Peter Shirley, Fundamentals of Computer Graphics, Peters, Wellesley
\item
  Andrew Woo, et al., OpenGL Programming Guide, Version 4.3,
  Addison-Wesley,
\item
  Tomas Akenine-Möller, Eric Haines, und Naty Hoffman, Real-Time
  Rendering, 3. Ausgabe, Peters, Wellesley
\item
  Randi J. Rost, John M. Kessenich, Barthold Lichtenbelt, OpenGL Shading
  Language, 2. Ausgabe, Addison-Wesley
\item
  Alan Watt, 3D Computer Graphics, Addison-Wesley
\item
  Frank Nielsen, Visual Computing, Charles River Media, 2005
\item
  James Foley, Andries Van Dam, et al., Computer Graphics : Principles
  and Practice, 2. Ausgabe, Addison-Wesley
\end{itemize}

\subsection*{Audiovisuelles Medienprojekt
2}\label{audiovisuelles-medienprojekt-2-3}

\begin{itemize}
\tightlist
\item
  James Monaco, Film verstehen, Rowolth Taschenbuch Verlag Hamburg,
  1980, ISBN 3-499-162717
\item
  Syd Field, Drehbuchschreiben für Film und Fernsehen, München 2003,
  ISBN 354836473X
\item
  Steven D. Katz, Die Richtige Einstellung, Zweitausendeins, Frankfurt
  a.M.1998,ISBN 3-86150-229-1
\item
  David Lewis Yewdall, Practical Art of Motion Picture Sound, Focal
  Press, USA 2003, ISBN 0-240-80525-9
\item
  Hans Kornacher \& Manfred Stross, Dokumentarisches Videofilmen,
  Augustus Verlag, Augsburg, 1992, ISBN 3-8043-5474-2
\item
  Hans Beller Hg., Handbuch der Filmmontage, München: TR-Verlagsunion,
  1993, ISBN 3-8058-2357-6
\item
  Karel Reisz, Gavin Millar, Geschichte und Technik der Filmmontage,
  München: Filmlandpresse, 1988, ISBN 3-88690-071-1
\item
  Chris Vogler, Die Reise des Drehbuchschreibens, Verlag Zweitausendeins
\item
  Wolfgang Lanzenberger, Michael Müller, Unternehmensfilme drehen:
  Business Movies im digitalen Zeitalter, ISBN 978-386764367
\end{itemize}

\subsection*{Visuelle Effekte und
Animation}\label{visuelle-effekte-und-animation-3}

\begin{itemize}
\tightlist
\item
  Flückiger Barbara, Visual Effects: Filmbilder aus dem Computer
  (Zürcher Filmstudien), Schüren Verlag GmbH, 2008, ISBN 978-3894725181
\item
  Bertram Sascha, VFX (Praxis Film), UVK, 2005, ISBN 978-3896695154
\end{itemize}

\subsection*{Cross-Platform Game Development mit Unity
3D}\label{cross-platform-game-development-mit-unity-3d-3}

\begin{itemize}
\tightlist
\item
  Unity 3D API (\url{https://docs.unity3d.com/ScriptReference/})
\item
  Unity 3D Manual (\url{https://docs.unity3d.com/Manual/index.html})
\item
  Unity Tutorials (\url{https://unity3d.com/de/learn/tutorials})
\item
  Ian Millington, John Funge,~ Artificial Intelligence For Games, Second
  Edition, CRC Press, 2009
\item
  Mat Buckland,~ Programming Game AI by Example, Wordware Game
  Developers Library, 2004
\item
  Steve Rabin et al.,~ AI Game Programming Wisdom 1-4, Cengage Learning
\end{itemize}

\subsection*{Prozedurale Generierung virtueller
Welten}\label{prozedurale-generierung-virtueller-welten-3}

\begin{itemize}
\tightlist
\item
  Peter Shirley, Steve Marschner, ``Fundamentals of Computer Graphics``,
  CRC Press
\item
  David Salomon, ``Curves and Surfaces for Computer Graphics'', Springer
\item
  Carsten Seifert, ``Spiele entwickeln mit Unity 5'', Hanser
\item
  Noor Shaker, Julian Togelius, Mark J. Nelson, ``Procedural Content
  Generation in Games (Computational Synthesis and Creative Systems)'',
  Springer
\item
  Ryan Watkins, ``Procedural Content Generation for Unity Game
  Development'', Packt Publishing
\item
  Dale Green, ``Procedural Content Generation for C++ Game
  Development'', Packt Publishing
\end{itemize}

\chapter{Web Development}\label{web-development}

\begin{modulHead}
\textbf{Modulverantwortlich}: Prof.~Dr.~Kristian
Fischer
\end{modulHead}
\begin{modulHead}
\textbf{Kürzel}:
WD
\end{modulHead}
\begin{modulHead}
\textbf{Studiensemester}:
4
\end{modulHead}
\begin{modulHead}
\textbf{Sprache}:
deutsch
\end{modulHead}
\begin{modulHead}
\textbf{Kreditpunkte}:
20
\end{modulHead}
\begin{modulHead}
\textbf{Voraussetzungen nach
Prüfungsordnung}: keine
\end{modulHead}
\begin{modulHead}
\textbf{Typ}:
Vertiefungsmodul
\end{modulHead}


\section*{Kurzbeschreibung}\label{kurzbeschreibung-5}

Einführungen in Konzepte, Techniken und Arbeitsweisen der Web
Entwicklung.

\section*{Lehrform/SWS:}\label{lehrformsws-26}

18 SWS: Vorlesung 6 SWS, Praktikum 6 SWS, Projekt 4 SWS

\section*{Lehrveranstaltungen}\label{lehrveranstaltungen-1}

\begin{itemize}
\tightlist
\item
  Web Frontend Entwicklung
\item
  Serverseitige Frameworks und Dienste
\item
  Internet of Things
\item
  Web Development Projekt
\end{itemize}

\section*{Arbeitsaufwand:}\label{arbeitsaufwand-25}

\begin{itemize}
\tightlist
\item
  Web-Frontend Entwicklung: Vorlesung, Seminar(50 Stunden);
  Selbstlernphase (100 Stunden)
\item
  Serverseitige Frameworks und Dienste: Vorlesung, Seminar(50 Stunden);
  Selbstlernphase (100 Stunden)
\item
  Internet of Things: Vorlesung, Seminar (50 Stunden); Selbstlernphase
  (100 Stunden)
\item
  Web Development Projekt: Projekt (150 Stunden)
\end{itemize}

\section*{Angestrebte
Lernergebnisse:}\label{angestrebte-lernergebnisse-26}

\subsection*{Web Frontend
Entwicklung:}\label{web-frontend-entwicklung}

Die Studierenden kennen wesentliche Konzepte und Technologien des
Web-Frontend Developments und können diese anwenden, um eigenständig im
Team Web-Frontends zu konzipieren, realisieren und optimieren.

Die Studierenden sind in der Lage ein gegebenes Gestaltungskonzept zu
verstehen und zu erweitern, um dies als Web-Frontend umzusetzen.

Die Studierenden kennen Web-Frontend Frameworks und sind in der Lage
diese kritisch zu beurteilen und auf Basis der Anforderungen eines
konkreten Projekts das optimale Framework Set zu konfektionieren und die
Auswahl zu begründen.

Die Studierenden kennen das Zusammenspiel von server- und clientseitigen
Komponenten im Bereich des Webs und können Web-Frontends konzipieren und
realisieren, die mit serverseitigen Komponenten und Diensten möglichst
optimal zusammen arbeiten. Sie können außerdem, bezogen auf eine
konkrete Aufgabenstellung, abwägen, welche Funktionalitäten clientseitig
und welche serverseitig gelöst werden sollten.

\subsection*{Serverseitige Frameworks und
Dienste:}\label{serverseitige-frameworks-und-dienste}

Die Studentinnen und Studenten kennen

\begin{itemize}
\tightlist
\item
  wesentliche Frameworks, Dienste und Werkzeuge für die serverseitige
  Entwicklung von Web Anwendungen
\item
  können ausgewählte Frameworks, Dienste und Tools in einem
  Projektkontext anwenden.
\end{itemize}

Die Kompetenz zur systematischen Entwicklung von Systemen in einem
arbeitsteiligen Team wird eingeübt und vertieft. Kenntnisse aus den
anderen Modulen der Vertiefung werden vertieft und verknüpft und im
Rahmen eines konkreten Projektauftrags angewendet.

Die Studierenden sind in der Lage ein Projektbriefing zu durchdringen
und daraus einen Projektauftrag abzuleiten und diesen im Team
abzuarbeiten.

Den Teilnehmern steht eine Auswahl an Techniken und Frameworks zur
Verfügung, aus dem sie die passenden Ansätze begründet auswählen und
anwenden können.

Die StudentenInnen sind in der Lage eine komplexe Anwendung im Web über
mehrere Endgeräte hinweg zu planen, zu realisieren und zu dokumentieren.

\subsection*{Internet of Things:}\label{internet-of-things}

In diesem Modul lernen die Teilnehmer das Gebiet Internet of Things
kennen. Dabei liegt ein besonderer Fokus auf der Bedeutung des Web für
Applikationen jenseits eines Browsers. Immer mehr Alltagsgegenstände
werden mit Technologien angereichert, die eine Dienste-Bereitstellung
oder Dienst-Nutzung über das Web ermöglichen (beispielsweise das Steuern
von Gegenständen oder das Erfassen von Sensordaten). In diesem Modul
werden relevante Konzepte und aktuelle Technologien für das Internet der
Dinge diskutiert und in prototypischen Anwendungen erprobt.

Studierende können nach diesem Modul selbstständig Anwendungen für das
Internet of Things konzipieren und realisieren, indem Sie

\begin{itemize}
\tightlist
\item
  hardware-nahe Aspekte im Design der Applikation berücksichtigen,
\item
  eine System-Architektur entwerfen,
\item
  kriterien-basiert geeignete Technologien zur Realisierung auswählen,
\item
  eine prototypische Anwendung implementieren,
\item
  inkrementell-iterativ vorgehen und Projektentscheidungen auf der
  Grundlage vorliegender Zwischenstände treffen.
\end{itemize}

Die Studierenden

\begin{itemize}
\tightlist
\item
  kennen ausgewählte Methoden und Frameworks für die Web Entwicklung im
  Front-End, im Back-End und in vernetzten Geräten (IoT),
\item
  können eine Methoden und Technologiewahl für einen Projektkontext
  fachlich begründen,
\item
  können Frameworks und Methoden zur Realisierung von Proof-of-Concepts
  in einem Projektkontext einsetzen und
\item
  können die erzielten Ergebnisse fachlich, kritisch einordnen und
  diskutieren,
\item
  um kompetent in Web Entwicklungs Teams mitwirken zu können.
\end{itemize}

\section*{Inhalt:}\label{inhalt-26}

\subsection*{Web Frontend
Entwicklung:}\label{web-frontend-entwicklung-1}

\begin{itemize}
\tightlist
\item
  Web Basics: HTML, CSS, Javascript
\item
  CSS: Komplexe Layouts \& Responsivität
\item
  Javascript: Dynamische Anwendungen
\item
  Media Types
\item
  CSS Frameworks
\item
  CSS Preprozessoren
\item
  Javascript Frameworks
\item
  Performance
\item
  Microdata, Internationalisierung, SEO, Barrierefreiheit
\end{itemize}

\subsection*{Serverseitige Frameworks und
Dienste:}\label{serverseitige-frameworks-und-dienste-1}

\begin{itemize}
\tightlist
\item
  NodeJS
\item
  Services im Web: Amazon WS (AWS), Google Firebase
\item
  NoSQL Datenbanken
\item
  Web Analyse: Piwik,
\end{itemize}

Ausgewählte Tools sollen tiefgreifend erarbeitet werden und in einem
Projektkontext angewendet werden. Dies erfolgt in der Regel in dem
begleitenden Projekt

\subsection*{Internet of Things:}\label{internet-of-things-1}

Zu Begin des Moduls findet eine Einführung durch den Dozenten statt.
Danach werden in seminaristischem Unterricht verschiedene aktuelle und
relevante Technologie des Internet of Things vermittelt, demonstriert
und erprobt (bspw. Raspberry Pi, Arduino, MQTT, Node-Red, Johnny Five,
AndroidThings). Diese werden in Teams zur Entwicklung eines eigenen
IoT-Prototypen eingesetzt.

Weitere Inhalte sind:

\begin{itemize}
\tightlist
\item
  Physical Computing
\item
  Prototyping und Retrofitting
\item
  Hardware (bspw. RaspberryPi und Arduino)
\item
  Sensoren und Aktoren
\item
  Frameworks (bspw. NodeRed und Johnny Five)
\item
  Architekturen und Protokolle (bspw. event-basierte Architekturen und
  MQTT)
\item
  Mobile Web- und Smartphone-Sensoren (bspw. GPS, Beacons)
\end{itemize}

\section*{Studien-/Prüfungsleistungen:}\label{studien-pruxfcfungsleistungen-24}

Projektarbeit mit Projektpräsentationsprüfung und Fachgespräch.

\section*{Medienformen:}\label{medienformen-17}

Beamergestützte Vorträge, Rechnergestützte Workshops

\section*{Literatur:}\label{literatur-22}

\begin{itemize}
\tightlist
\item
  Randy Connolly, Ricardo Hoar: Fundamentals of Web Development
\item
  Andy Clark: Hardboiled Web-Design
\item
  Tilkov et al.: REST und HTTP- Entwicklung und Integration nach dem
  Architekturstil des Web, dpunkt.verlag 2015
\item
  Watkin: Practical XMPP, Packt Publishing 2016
\item
  Saint-Andre: XMPP: THe Definitive Guide, OReilly 2009
\item
  Roy: RabbitMQ in Depth, Manning 2016
\item
  Newman: Building Microservices: Designing fine-grained systems,
  OReilly 2015
\end{itemize}

\chapter{Social Computing}\label{social-computing}

\begin{modulHead}
\textbf{Modulverantwortlich}: Prof.~Dr.~Christian
Kohls
\end{modulHead}
\begin{modulHead}
\textbf{Kürzel}:
SC
\end{modulHead}
\begin{modulHead}
\textbf{Studiensemester}:
4
\end{modulHead}
\begin{modulHead}
\textbf{Sprache}:
deutsch
\end{modulHead}
\begin{modulHead}
\textbf{Kreditpunkte}:
20
\end{modulHead}
\begin{modulHead}
\textbf{Voraussetzungen nach
Prüfungsordnung}: keine
\end{modulHead}
\begin{modulHead}
\textbf{Typ}:
Vertiefungsmodul
\end{modulHead}


\section*{Kurzbeschreibung}\label{kurzbeschreibung-6}

In der Vertiefung „Social Computing'' werden die Wechselwirkungen
zwischen Gesellschaft und Informatik in den Mittelpunkt gestellt.
Rechnersysteme und Netzwerke werden von Menschen intentional gestaltet,
ausgerichtet an gesellschaftlichen Normen, Prozessen und Bedürfnissen.
Gleichzeitig beeinflussen IT-Systeme diese gesellschaftlichen Normen und
verändern Prozesse in allen Lebensbereichen. Die verantwortungsbewusste
Konzeption und Realisierung von soziotechnischen Systemen (z.B. Social
Software, Online Communities, e-Health, e-Government und e-Learning
Angebote) sowie die empirische Evaluation existierender Systeme sind
zentrale Ziele. Lösungen sollen unter ganzheitlichen Gesichtspunkten
entwickelt werden. Verschiedene Wertvorstellungen und Interessen
unterschiedlicher Stakeholder müssen identifiziert und berücksichtig
werden.

Der Schwerpunkt verbindet daher Theorien, Modelle und Methodik der
Human- und Sozialwissenschaften mit anwendungsorientierter Informatik.
Studierende sollen in der Lage sein, computergestützte Systeme nach
ethischen, politischen, sozialen und psychologischen Kriterien zu
bewerten, zu planen und umsetzen zu können.

Ziel ist es, soziale Innovation durch digitale Anwendungen entstehen zu
lassen. Neben den empirischen Methoden werden Designmethoden vermittelt,
sowohl auf der konzeptionellen als auch auf der softwaretechnischen
Implementierungsebene, um robuste, sichere und flexible Systeme zu
gestalten.

Das Modul ist aus vier Teilbereichen aufgebaut, von denen zwei
verpflichtend sind und zwei weitere aus einem Wahlkatalog gewählt werden
können.

Verpflichtende Kurse im Umfang von je 5 CP sind:

\begin{itemize}
\tightlist
\item
  Soziotechnische Systeme
\item
  Empirische Forschungsmethoden
\item
  Gamification
\item
  Projekt
\end{itemize}

Die Kurse sind in der Regel projektbasiert aufgebaut, so dass sowohl
theoretischer Hintergrund als auch praxisnahes Wissen vermittelt wird
und zur Anwendung kommt.

\section*{Lehrform/SWS:}\label{lehrformsws-27}

Vorlesung, Praktikum / Projekt, Übung mit kursabhängigen Schwerpunkten

\section*{Arbeitsaufwand:}\label{arbeitsaufwand-26}

Gesamtaufwand 600 h, davon

\begin{itemize}
\tightlist
\item
  150h Soziotechnische Systeme
\item
  150h Empirische Forschungsmethoden
\item
  150h Gamification
\item
  150h Projekt
\end{itemize}

\section*{Angestrebte
Lernergebnisse:}\label{angestrebte-lernergebnisse-27}

siehe jeweilige Fächerbeschreibung

\section*{Inhalt:}\label{inhalt-27}

siehe Modulbeschreibungen:

\begin{itemize}
\tightlist
\item
  Soziotechnische Systeme
\item
  Empirische Forschungsmethoden
\item
  Gamification
\item
  Projekt
\end{itemize}

\section*{Studien-/Prüfungsleistungen:}\label{studien-pruxfcfungsleistungen-25}

Teilprüfungen in den jeweiligen Kursen.

\section*{Medienformen:}\label{medienformen-18}

siehe jeweilige Kursbeschreibung

\section*{Literatur:}\label{literatur-23}

siehe jeweilige Kursbeschreibung

\chapter{Wahlpflichtmodul}\label{wahlpflichtmodul}

\begin{modulHead}
\textbf{Modulverantwortlich}: alle Informatik
Professoren
\end{modulHead}
\begin{modulHead}
\textbf{Kürzel}:
WPF
\end{modulHead}
\begin{modulHead}
\textbf{Studiensemester}:
5
\end{modulHead}
\begin{modulHead}
\textbf{Sprache}:
deutsch
\end{modulHead}
\begin{modulHead}
\textbf{Kreditpunkte}:
5
\end{modulHead}
\begin{modulHead}
\textbf{Voraussetzungen nach
Prüfungsordnung}: keine
\end{modulHead}
\begin{modulHead}
\textbf{Typ}:
Pflichtmodul
\end{modulHead}


\section*{Lehrform/SWS:}\label{lehrformsws-28}

Je nach Modul

\section*{Arbeitsaufwand:}\label{arbeitsaufwand-27}

150 Stunden

\section*{Angestrebte
Lernergebnisse:}\label{angestrebte-lernergebnisse-28}

Fachliche Vertiefung oder Verbreiterung, nach persönlichem Interesse. Es
kann eines der Module aus dem Katalog aller Module der Informatik
Bachelorstudiengänge gewählt werden. Auch Pflichtmodule anderer
Informatik Studiengänge am Campus können als Wahlpflichtmodule in der
Medieninformatik belegt werden.

\section*{Inhalt:}\label{inhalt-28}

Je nach Modul

\section*{Studien-/Prüfungsleistungen:}\label{studien-pruxfcfungsleistungen-26}

Je nach Modul

\section*{Medienformen:}\label{medienformen-19}

Je nach Modul

\section*{Literatur:}\label{literatur-24}

Je nach Modul
